\def\iangle{35} % Angle of the inclined plane
\def\down{0}
\def\arcr{0.7cm} % Radius of the arc used to indicate angles
\newcommand\centerofmass{%
    \tikz[radius=0.2em, red] {%
        \fill (0,0) -- ++(0.2em,0) arc [start angle=0,end angle=90] -- ++(0,-0.4em) arc [start angle=270, end angle=180];%
        \draw (0,0) circle;%
    }%
}

\begin{tikzpicture}[
    force/.style={>=latex,draw=blue,fill=blue},
    axis/.style={densely dashed,gray,font=\small},
    M/.style={rectangle,draw,fill=lightgray,minimum size=0.7cm,thin},
    m/.style={rectangle,draw=black,fill=lightgray,minimum size=0.3cm,thin},
    plane/.style={draw=black,fill=blue!10},
    string/.style={draw=red, thick},
    pulley/.style={thick},
    wheel/.style={fill=black, rounded corners=1.5pt},
]

    \begin{scope}
        \node[draw,anchor=south west,pattern=north west lines,minimum width=2cm,minimum height=1cm] (Ob1) at (0,2) {};
        \node[draw,anchor=south west,pattern=north west lines,minimum width=1cm,minimum height=1cm] (Ob2) at (2.5,-1.5) {};
    \end{scope}

     \begin{scope}[rotate=0]
        \node[M,transform shape] (M1) at (-2,-1) {\centerofmass};
        % Draw axes and help lines
        % {[axis,->]
        %     \draw (M1) -- ++(0,2) node(y1_axis)[right] {$y'$};
        % }
        % Forces
        {[force,->]
            % Assuming that Mg = 1. The normal force will therefore be cos(alpha)
            \draw (M1.east) -- ++(1,0) node[above, blue] {};
        }
        \draw[wheel, fill=gray] (M1.south west) rectangle ++(.4,-.1) node[]{};
        \draw[wheel, fill=gray] (M1.north west) rectangle ++(.4,.1)  node[]{};

        \node[] at (-2,-2) {$\textcolor{red}{\mathbf{P}_{t}}$};
    \end{scope}

    \onslide<1->


    \draw[, ->] (-2,-1) -- ++(2,0) node[below] {$x$};
    \draw[, ->] (-2,-1) -- ++(0,2) node(y_axis)[right] {$y$};
    \draw[gray, ->] (-2,-1) -- ++(-.5,-.5) node[left] {$z$};

    \onslide<2>

        % lidar
        \draw[axis, gray] (M1.center) -- (Ob1.west) node[] {x};
        \draw[axis, gray] (M1.center) -- (Ob1.south west) node[] {x};
        \draw[axis, gray] (M1.center) -- (Ob1.south) node[] {x};
        \draw[axis, gray] (M1.center) -- (Ob1.south east) node[] {x};
        %%
        \draw[axis, gray] (M1.center) -- (Ob2.north west) node[] {x};
        \draw[axis, gray] (M1.center) -- (Ob2.west) node[] {x};
        \draw[axis, gray] (M1.center) -- (Ob2.south west) node[] {x};
        \draw[axis, gray] (M1.center) -- (Ob2.north east) node[] {x};

    \onslide<3->
    %% Free body diagram of M
    \begin{scope}[rotate=\iangle]
        \node[M,transform shape] (M2) at (1,-0.5) {\centerofmass};
        % Draw axes and help lines
        {[axis,->]
            % \draw (M2) -- ++(0,2) node(y1_axis)[right] {};
            % \draw (M2) -- ++(2,0) node[right] {};
            % Indicate angle. The code is a bit awkward.
            \draw[solid,shorten >=0.5pt, xshift=30, yshift=-15] (\down-\iangle:\arcr)
                arc(\down-\iangle:\down:\arcr);
            \node[xshift=35, yshift=2] at (\down-0.5*\iangle:1.3*\arcr) {$\theta$};
        }
        % Forces
        {[force,->]
            % Assuming that Mg = 1. The normal force will therefore be cos(alpha)
            \draw (M2.east) -- ++(1,0) node[above, blue] {};
        }
        \draw[wheel] (M2.south west) rectangle ++(.4,-.1) node[below]{};
        \draw[wheel] (M2.north west) rectangle ++(.4,.1)  node[left]{};
        \node[] at (0.5,0.5) {$\textcolor{blue}{\mathbf{P}_{t+1}}$};
    \end{scope}
    % Draw gravity force. The code is put outside the rotated
    % scope for simplicity. No need to do any angle calculations. 
    \draw[axis,] (M2.center) -- ++(1,0) node[below] {};

    \onslide<4>

    % lidar
    \draw[red] (M2.center) -- (Ob1.south west) node[] {x};
    \draw[red] (M2.center) -- (Ob1.south) node[] {x};
    \draw[red] (M2.center) -- (Ob1.south east) node[] {x};
    %%
    \draw[red] (M2.center) -- (Ob2.north west) node[] {x};
    \draw[red] (M2.center) -- (Ob2.west) node[] {x};
    \draw[red] (M2.center) -- (Ob2.south west) node[] {x};
    \draw[red] (M2.center) -- (Ob2.north east) node[] {x};



    % \node[right, gray,font=\small, xshift=8] at (M1.center) {$\{B\}$};
    % %%
    % \node[left, gray,font=\small, xshift=-10] at (0,0) {$\{A\}$};
\end{tikzpicture}
