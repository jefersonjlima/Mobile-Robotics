\documentclass[aspectratio=169]{beamer}
\usetheme{simple}
\usepackage[english]{babel}
\usepackage[utf8]{inputenc} 
\usepackage{lmodern}
\usepackage{ragged2e}
\usefonttheme[onlymath]{serif} 
\usepackage[scale=2]{ccicons} 
% \setbeamertemplate{caption}[numbered]
\usepackage{copyrightbox}

\usepackage{graphicx,hyperref,url,pgfplots}
\usepackage{amsmath} 
\usepackage{array,booktabs}
\pgfplotsset{compat=1.13}
\usepackage{bibentry}
\usepackage[alf,abnt-etal-list=0,abnt-etal-cite=2]{abntex2cite}
\usepackage[normalem]{ulem}

\usepackage[
    type={CC},
    modifier={by-nc-sa},
    version={4.0},
]{doclicense}

\setbeamercovered{invisible} 
% \newcommand{\pausar}{\pause}
\newcommand{\pausar}{\pause}
\newcommand{\df}[1]{\,\mathrm{d}#1}
\newcommand{\parcial}[3]{\dfrac{\partial^{#1}#2}{\partial #3^{#1}}}
\newcommand{\cpright}[2]{\copyrightbox[b]{#1}{\tiny Source: #2}}

\usepackage{tikz}
\usetikzlibrary{automata,positioning}
\usepackage{xcolor}
\usetikzlibrary{scopes}
\usepackage{verbatim}
\usetikzlibrary{patterns}

\usepackage{listings}
  \lstdefinestyle{ascii-tree}{
    literate={├}{|}1 {─}{--}1 {└}{+}1 
  }
	\definecolor{codegreen}{rgb}{0,0.6,0}
	\definecolor{codegray}{rgb}{0.5,0.5,0.5}
	\definecolor{codepurple}{rgb}{0.58,0,0.82}
	\definecolor{backcolour}{rgb}{0.92,0.92,0.92}
	\lstset{language=Python, 
	backgroundcolor=\color{backcolour},   
	commentstyle=\color{codegreen},
	keywordstyle=\color{magenta},
	numberstyle=\tiny\color{codegray},
	stringstyle=\color{codepurple},
	basicstyle=\fontsize{8}{11}\ttfamily,
	frame=lines,
%	numbers=left,
	tabsize=2,
	morekeywords={models, lambda, forms},
	showstringspaces=false}


% --------------------------------------------------------------------------------------------

\title{Mobile Robots}
\subtitle{Pose Estimation - ICP Algorithm}
\date{\today}
\author[Jeferson José de Lima]{
  \textbf{Professor}: Jeferson José de Lima}
\institute{Academic Department of Informatics (DAINF) \\ Federal University of Technology - Paraná (UTFPR) at Pato Branco, PR, Brazil}

\begin{document}
\maketitle
\justify


\begin{frame}{Useful Information}

	\begin{block}{License}
        \doclicenseThis
    \end{block}

	\begin{block}{links:}
		\begin{enumerate}
			\item \href{https://gitlab.com/cursoseaulas/robotica-movel/-/wikis/home}{Mobile Robots - Gitlab Page}
		\end{enumerate}
	\end{block}
\end{frame}




\begin{frame}{Pose Estimation Lesson}
	\framesubtitle{}
	After this lesson, you will able to:
	\begin{itemize}
		\item Describe the point LIDAR points and how it can be used for state estimation
		\item Implement the Iterative Closest Point (ICP) algorithm
	\end{itemize}
\end{frame}


\begin{frame}{ROS Philosophy}
	\framesubtitle{}
		
    \begin{columns}
        \begin{column}[c]{0.5\textwidth}
            \def\iangle{35} % Angle of the inclined plane
\def\down{0}
\def\arcr{0.7cm} % Radius of the arc used to indicate angles
\newcommand\centerofmass{%
    \tikz[radius=0.2em] {%
        \fill (0,0) -- ++(0.2em,0) arc [start angle=0,end angle=90] -- ++(0,-0.4em) arc [start angle=270, end angle=180];%
        \draw (0,0) circle;%
    }%
}

\begin{tikzpicture}[
    force/.style={>=latex,draw=blue,fill=blue},
    axis/.style={densely dashed,gray,font=\small},
    M/.style={rectangle,draw,fill=lightgray,minimum size=0.7cm,thin},
    m/.style={rectangle,draw=black,fill=lightgray,minimum size=0.3cm,thin},
    plane/.style={draw=black,fill=blue!10},
    string/.style={draw=red, thick},
    pulley/.style={thick},
    wheel/.style={fill=black, rounded corners=1.5pt},
]
    %% Free body diagram of M
    \begin{scope}[rotate=\iangle]
        \node[M,transform shape] (M) {\centerofmass};
        % Draw axes and help lines
        {[axis,->]
            \draw (M) -- ++(0,2) node(y1_axis)[right] {$y_M$};
            \draw (M) -- ++(2,0) node[right] {$x_M$};
            % Indicate angle. The code is a bit awkward.
            \draw[solid,shorten >=0.5pt] (\down-\iangle:\arcr)
                arc(\down-\iangle:\down:\arcr);
            \node at (\down-0.5*\iangle:1.3*\arcr) {$\phi_M$};
        }
        % Forces
        {[force,->]
            % Assuming that Mg = 1. The normal force will therefore be cos(alpha)
            \draw (M.east) -- ++(1,0) node[above, blue] {$v_M$};
        }
        \draw[wheel] (M.south west) rectangle ++(.4,-.1) node[below]{$v_{M_R}$};
        \draw[wheel] (M.north west) rectangle ++(.4,.1)  node[left]{$v_{M_L}$};
    \end{scope}
    % Draw gravity force. The code is put outside the rotated
    % scope for simplicity. No need to do any angle calculations. 
    \draw[axis,] (M.center) -- ++(1,0) node[below] {};
    %%
    \node[right, gray,font=\small, xshift=8] at (y1_axis) {$\{M\}$};
    %%
    \draw[, ->] (-2,-1) -- ++(4,0) node[below] {$x_I$};
    \draw[, ->] (-2,-1) -- ++(0,3) node(y_axis)[right] {$y_I$};
    \draw[gray, ->] (-2,-1) -- ++(-.5,-.5) node[left] {$z_I$};
    \node[left, gray,font=\small, xshift=-10] at (y_axis) {$\{I\}$};
\end{tikzpicture}

        \end{column}
        \begin{column}[c]{0.5\textwidth}
            \def\iangle{35} % Angle of the inclined plane
\def\down{0}
\def\arcr{0.7cm} % Radius of the arc used to indicate angles
\newcommand\centerofmass{%
    \tikz[radius=0.2em] {%
        \fill (0,0) -- ++(0.2em,0) arc [start angle=0,end angle=90] -- ++(0,-0.4em) arc [start angle=270, end angle=180];%
        \draw (0,0) circle;%
    }%
}

\begin{tikzpicture}[
    force/.style={>=latex,draw=blue,fill=blue},
    axis/.style={densely dashed,gray,font=\small},
    M/.style={rectangle,draw,fill=lightgray,minimum size=0.7cm,thin},
    m/.style={rectangle,draw=black,fill=lightgray,minimum size=0.3cm,thin},
    plane/.style={draw=black,fill=blue!10},
    string/.style={draw=red, thick},
    pulley/.style={thick},
    wheel/.style={fill=black, rounded corners=1.5pt},
]
    %% Free body diagram of M
    \begin{scope}[rotate=\iangle]
        \node[] (M) {};
%        \node[below, purple] at (M) {${}^B_A\mathbf{P}$};
        % Draw axes and help lines
        {[axis,->]
            \draw (M.center) -- ++(0,2) node(y1_axis)[right] {$y_M$};
            \draw (M.center) -- ++(2,0) node[right] {$x_M$};
            % Indicate angle. The code is a bit awkward.
            \draw[solid,shorten >=0.5pt] (\down-\iangle:\arcr)
                arc(\down-\iangle:\down:\arcr);
            \node[xshift=10, brown]at (\down-0.5*\iangle:1.3*\arcr) {$\mathbf{R}_z(\phi)$};
        }
        % Forces
        {[force,->]
            % Assuming that Mg = 1. The normal force will therefore be cos(alpha)
            \draw (M.center) -- ++(1,0) node[above, blue] {$v_M$};
        }
    \end{scope}
    % Draw gravity force. The code is put outside the rotated
    % scope for simplicity. No need to do any angle calculations. 
    \draw[axis,] (M.center) -- ++(1,0) node[below] {};
    %%
    \node[right, gray,font=\small, xshift=8] at (y1_axis) {$\{M\}$};
    %%
    \draw[, ->] (-2,-1) -- ++(4,0) node[below] {$x_I$};
    \draw[, ->] (-2,-1) -- ++(0,3) node(y_axis)[right] {$y_I$};
    \draw[gray, ->] (-2,-1) -- ++(-.5,-.5) node[left] {$z_I$};
    \node[left, gray,font=\small, xshift=-10] at (y_axis) {$\{I\}$};
    \draw [densely dashed,red,] (-2,-1)-- (M.center) node[above, midway] {${}^A\mathbf{Q}$};
\end{tikzpicture}

  
        \end{column}
    \end{columns}

\end{frame}



\begin{frame}[t, allowframebreaks]
	\frametitle{Referências}
	\bibliography{../references.bib}
\end{frame}

\end{document}