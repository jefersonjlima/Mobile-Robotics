\documentclass[aspectratio=169]{beamer}
\usetheme{simple}
\usepackage[english]{babel}
\usepackage[utf8]{inputenc} 
\usepackage{lmodern}
\usepackage{ragged2e}
\usefonttheme[onlymath]{serif} 
\usepackage[scale=2]{ccicons} 
% \setbeamertemplate{caption}[numbered]
\usepackage{copyrightbox}

\usepackage{graphicx,hyperref,url,pgfplots}
\usepackage{amsmath} 
\usepackage{array,booktabs}
\pgfplotsset{compat=1.13}
\usepackage{bibentry}
\usepackage[alf,abnt-etal-list=0,abnt-etal-cite=2]{abntex2cite}
\usepackage[normalem]{ulem}

\usepackage[
    type={CC},
    modifier={by-nc-sa},
    version={4.0},
]{doclicense}

\setbeamercovered{invisible} 
% \newcommand{\pausar}{\pause}
\newcommand{\pausar}{\pause}
\newcommand{\df}[1]{\,\mathrm{d}#1}
\newcommand{\parcial}[3]{\dfrac{\partial^{#1}#2}{\partial #3^{#1}}}
\newcommand{\cpright}[2]{\copyrightbox[b]{#1}{\tiny Source: #2}}

\usepackage{tikz}
\usetikzlibrary{automata,positioning}
\usepackage{xcolor}
\usetikzlibrary{scopes}
\usepackage{verbatim}
\usetikzlibrary{patterns}

\usepackage{listings}
  \lstdefinestyle{ascii-tree}{
    literate={├}{|}1 {─}{--}1 {└}{+}1 
  }
	\definecolor{codegreen}{rgb}{0,0.6,0}
	\definecolor{codegray}{rgb}{0.5,0.5,0.5}
	\definecolor{codepurple}{rgb}{0.58,0,0.82}
	\definecolor{backcolour}{rgb}{0.92,0.92,0.92}
	\lstset{language=Python, 
	backgroundcolor=\color{backcolour},   
	commentstyle=\color{codegreen},
	keywordstyle=\color{magenta},
	numberstyle=\tiny\color{codegray},
	stringstyle=\color{codepurple},
	basicstyle=\fontsize{8}{11}\ttfamily,
	frame=lines,
%	numbers=left,
	tabsize=2,
	morekeywords={models, lambda, forms},
	showstringspaces=false}


% --------------------------------------------------------------------------------------------

\title{Mobile Robots}
\subtitle{Pose Estimation - ICP Algorithm}
\date{\today}
\author[Jeferson José de Lima]{
  \textbf{Professor}: Jeferson José de Lima}
\institute{Academic Department of Informatics (DAINF) \\ Federal University of Technology - Paraná (UTFPR) at Pato Branco, PR, Brazil}

\begin{document}
\maketitle
\justify


\begin{frame}{Useful Information}

	\begin{block}{License}
        \doclicenseThis
    \end{block}

	\begin{block}{links:}
		\begin{enumerate}
			\item \href{https://gitlab.com/cursoseaulas/robotica-movel/-/wikis/home}{Mobile Robots - Gitlab Page}
		\end{enumerate}
	\end{block}
\end{frame}




\begin{frame}{Pose Estimation Lesson}
	\framesubtitle{}
	After this lesson, you will able to:
	\begin{itemize}
		\item Describe the point LIDAR points and how it can be used for state estimation
		\item Implement the Iterative Closest Point (ICP) algorithm
	\end{itemize}
\end{frame}


\begin{frame}{ROS Philosophy}
	\framesubtitle{}
		
    \begin{columns}
        \begin{column}[c]{0.5\textwidth}
            \input{./images/car_1a_image.tex}
        \end{column}
        \begin{column}[c]{0.5\textwidth}
            \input{./images/car_2a_image.tex}
        \end{column}
    \end{columns}

\end{frame}



\begin{frame}[t, allowframebreaks]
	\frametitle{Referências}
	\bibliography{../references.bib}
\end{frame}

\end{document}