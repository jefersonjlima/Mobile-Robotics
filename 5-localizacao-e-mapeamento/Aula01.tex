\documentclass[t]{beamer}

\usepackage[brazil]{babel} 
\usepackage[utf8]{inputenc}
\usepackage{graphicx,hyperref,url,pgfplots}
\usetheme{Warsaw}
\usecolortheme{crane}
\usefonttheme[onlymath]{serif}  %amarelo
% previsualização animacao
\setbeamercovered{transparent}
% texto justificado
\usepackage{ragged2e}
\justifying

\pgfplotsset{compat=1.10}
\usepackage{amsmath} 
\usepackage{array,booktabs}
\usepackage[overload]{textcase}

% \usepackage{mcode}              % código matlab

\newcommand{\pausar}{\pause}
\newcommand{\pratica}{
\begin{frame}[c]
\textit{prática ...}
\begin{center}
\includegraphics[width=0.5\textwidth]{images/prog.jpg}
\end{center}
\end{frame}
}
\newcommand{\titleSlide}[1]{\textbf{\MakeUppercase{#1}}}

% paginas numero 
\addtobeamertemplate{navigation symbols}{}{%
    \usebeamerfont{footline}%
    \usebeamercolor[fg]{footline}%
    \hspace{1em}%
    \insertframenumber/\inserttotalframenumber
}

% \lstloadlanguages{C}
% \usepackage{listings}
% \usepackage{color}
% \usepackage{textcomp}
% \definecolor{listinggray}{gray}{0.9}
% \definecolor{lbcolor}{rgb}{0.9,0.9,0.9}
% \lstset{
% %	backgroundcolor=\color{lbcolor},
% 	tabsize=4,
% 	rulecolor=,
% 	language=C,
% 	    basicstyle=\ttfamily\scriptsize,
% %        basicstyle=\normalsize,
%         upquote=true,
%         aboveskip={1.5\baselineskip},
%         columns=fixed,
%         showstringspaces=false,
%         extendedchars=true,
%         breaklines=true,
%         prebreak = \raisebox{0ex}[0ex][0ex]{\ensuremath{\hookleftarrow}},
% %        frame=single,
%         showtabs=false,
%         showspaces=false,
%         showstringspaces=false,
%         identifierstyle=\ttfamily,
%         keywordstyle=\color[rgb]{0,0,1},
%         commentstyle=\color[rgb]{0.133,0.545,0.133},
%         stringstyle=\color[rgb]{0.627,0.126,0.941},
% }
% \lstloadlanguages{C}



% The title of the presentation:
%  - first a short version which is visible at the bottom of each slide;
%  - second the full title shown on the title slide;
\logo{\includegraphics[scale=0.1]{images/logo.jpg}}
\newcommand{\nologo}{\setbeamertemplate{logo}{}} % command to set the logo to nothing

\title[SLAM]{
  SLAM - \textit{Simultaneous Localization and Mapping}}

% Optional: a subtitle to be dispalyed on the title slide
%\subtitle{Teoria e Prática}

% The author(s) of the presentation:
%  - again first a short version to be displayed at the bottom;
%  - next the full list of authors, which may include contact information;
\author[Prof. Jeferson José de Lima]{
  \textbf{Professor}: Jeferson José de Lima\\  \medskip
  {\url{jefersonlima@utfpr.edu.br}}}
  

% The institute:
%  - to start the name of the university as displayed on the top of each slide
%    this can be adjusted such that you can also create a Dutch version
%  - next the institute information as displayed on the title slide
\institute[UTFPR-PB]{Departamento de Informática (DAINF-PB)}

% Add a date and possibly the name of the event to the slides
%  - again first a short version to be shown at the bottom of each slide
%  - second the full date and event name for the title slide
\date[2019.2]

\begin{document}

\begin{frame}
  \titlepage
\end{frame}

\begin{frame}
  \frametitle{Sumário}
  {\small {\small \tableofcontents}}
\end{frame}

\section{Informações Úteis}
  \frametitle{Informações Úteis}
\begin{frame}

\begin{enumerate}
\item Material disponível em: \href{https://gitlab.com/cursoseaulas/curso-arduino}{https://gitlab.com/cursoseaulas/curso-arduino} 
\end{enumerate}

\end{frame}

\section{Coordenadas Homogêneas}
%-----------------------------------------------------------------
\subsection{Definição}
\begin{frame}
  \href{http://www.joinville.ifsc.edu.br/~michael.klug/ROB74/Aulas/aula2_trans_geometricas.pdf
  }{material}
\end{frame}
%-----------------------------------------------------------------
\section{Teorema de Bayes}
\begin{frame}
  \titleSlide{Estimativa de Estados}

  \begin{itemize}
    \item Estado Estimado $x$ de um sistema observado $z$ e com controle em $u$.
    \item Objetivo:
  \end{itemize}

  \begin{equation}
    p(x|z,u)
  \end{equation}

\end{frame}
%-----------------------------------------------------------------

\subsection{Definição}
\begin{frame}
  \href{http://www.joinville.ifsc.edu.br/~michael.klug/ROB74/Aulas/aula2_trans_geometricas.pdf
  }{material}
\end{frame}
%-----------------------------------------------------------------


%-----------------------------------------------------------------
\section{Kalman Filter}
\subsection{Introdução}
\begin{frame}
  teste
\end{frame}
%-----------------------------------------------------------------
\begin{frame}
  \titleSlide{Modelos Lineares}
 
  \begin{itemize}
    \item Assumindo que o modelo é linear;
    \item Com ruído Gaussiano ()
  \end{itemize}

  \begin{align} 
    x_t &= A_t x_{t-1} + B u_t + \epsilon_t\\ 
    z_t &= C_t x_t + \delta_t
    \end{align}

    $A_t$ Matriz $(n \times n)$ que descreve os estados do modelo.

    $B_t$ Matriz $(n \times l)$ que descreve os estados do controle.

    $C_t$ Matrix $(k\times n)$ sendo os estados de $x_t$

    $\epsilon_t$ Variável aleatória do processo.

    $\delta_t$ Rúido aleatório com distribuição normal e covariância de $R_t$ e $Q_t$ respectivamente.

\end{frame}
%-----------------------------------------------------------------
\subsection{KF, EKF e UKF}
\begin{frame}
  teste
\end{frame}
%-----------------------------------------------------------------
\begin{frame}[allowframebreaks]

\textbf{Referências}

\begin{itemize}
\begin{small}
\item http://www.embarcados.com.br/arduino-entradasaidas-digitais/
\item \href{https://www.youtube.com/watch?v=DE6Jn2cB4J4&list=PLgnQpQtFTOGQrZ4O5QzbIHgl3b1JHimN_&index=5}{aula kalman} 
\item \href{https://www.youtube.com/watch?v=QZ5q59H2qaI}{Fusion Odometry and IMU}
\end{small}
\end{itemize}





%\small{\textbf{Referências}
%        \bibliographystyle{amsalpha}
%        \bibliography{reflatex.bib}}
\end{frame}
%-----------------------------------------------------------------
\end{document}
