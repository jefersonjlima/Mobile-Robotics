\documentclass[aspectratio=169]{beamer}
\usetheme{simple}
\usepackage[english]{babel}
\usepackage[utf8]{inputenc} 
\usepackage{lmodern}
\usepackage{ragged2e}
\usefonttheme[onlymath]{serif} 
\usepackage[scale=2]{ccicons} 
% \setbeamertemplate{caption}[numbered]
\usepackage{copyrightbox}
\DeclareMathOperator*{\argmin}{arg\,min}


\usepackage{graphicx,hyperref,url,pgfplots}
\usepackage{amsmath} 
\usepackage{array,booktabs}
\pgfplotsset{compat=1.13}
\usepackage{bibentry}
\usepackage[alf,abnt-etal-list=0,abnt-etal-cite=2]{abntex2cite}
\usepackage[normalem]{ulem}
\PassOptionsToPackage{demo}{graphicx}
\def\HiLi{\leavevmode\rlap{\hbox to \hsize{\color{yellow!50}\leaders\hrule height .8\baselineskip depth .5ex\hfill}}}

\usepackage[
    type={CC},
    modifier={by-nc-sa},
    version={4.0},
]{doclicense}

\setbeamercovered{invisible} 
\newcommand{\df}[1]{\,\mathrm{d}#1}
\newcommand{\parcial}[3]{\dfrac{\partial^{#1}#2}{\partial #3^{#1}}}
\newcommand{\cpright}[2]{\copyrightbox[b]{#1}{\tiny Source: #2}}
%Tikz
\newcommand{\tikzRot}{0}
\newcommand{\tikzTrans}{(0,0)}
\newcommand{\tikzShowrobot}{0}
\newcommand{\tikzOneCenter}{0}

\usepackage{tikz}
\usetikzlibrary{automata,positioning}
\usepackage{xcolor}
\usetikzlibrary{scopes}
\usepackage{verbatim}
\usetikzlibrary{patterns}
\usepackage{algorithm}
\usepackage{algpseudocode}
\def\NoNumber#1{{\def\alglinenumber##1{}\State #1}\addtocounter{ALG@line}{-1}}

\usepackage{listings}
  \lstdefinestyle{ascii-tree}{
    literate={├}{|}1 {─}{--}1 {└}{+}1 
  }
	\definecolor{codegreen}{rgb}{0,0.6,0}
	\definecolor{codegray}{rgb}{0.5,0.5,0.5}
	\definecolor{codepurple}{rgb}{0.58,0,0.82}
	\definecolor{backcolour}{rgb}{0.92,0.92,0.92}
	\lstset{language=Python, 
	backgroundcolor=\color{backcolour},   
	commentstyle=\color{codegreen},
	keywordstyle=\color{magenta},
	numberstyle=\tiny\color{codegray},
	stringstyle=\color{codepurple},
	basicstyle=\fontsize{8}{11}\ttfamily,
	frame=lines,
%	numbers=left,
	tabsize=2,
	morekeywords={models, lambda, forms},
	showstringspaces=false}

\usepackage{catchfile}
\newcommand{\getenv}[2][]{%
	\CatchFileEdef{\temp}{"|kpsewhich --var-value #2"}{\endlinechar=-1}%
	\if\relax\detokenize{#1}\relax\temp\else\let#1\temp\fi}
\getenv[\RMHOME]{RM_HOME_PATH}


% --------------------------------------------------------------------------------------------

\title{Mobile Robots}
\subtitle{outliers - Random Sample Consensus (RANSAC)
}
\date{\today}
\author[Jeferson José de Lima]{
  \textbf{Professor}: Jeferson José de Lima}
\institute{Academic Department of Informatics (DAINF) \\ Federal University of Technology - Paraná (UTFPR) at Pato Branco, PR, Brazil}

\begin{document}
\maketitle
\justify

\begin{frame}{Useful Information}

	\begin{block}{License}
        \doclicenseThis
    \end{block}

	\begin{block}{links:}
		\begin{enumerate}
			\item \href{https://moodle.utfpr.edu.br/course/view.php?id=14218}{Moodle}
			\item \href{https://gitlab.com/cursoseaulas/robotica-movel/-/wikis/home}{Mobile Robots - Gitlab Page}
		\end{enumerate}
	\end{block}
\end{frame}


\begin{frame}[fragile, t]{Pose Estimation}
	\framesubtitle{Random Sample Consensus (RANSAC)}
	\begin{minipage}[t]{0.5\textwidth}
 
		\begin{itemize}
			\item RANSAC (Random Sample Consensus) is a simple algorithm for robust fitting of models
			in the presence of many data outliers \cite{nuchter20083d}.
		\end{itemize}
	\end{minipage}
	\begin{minipage}[t]{0.5\textwidth}
		\begin{figure}
			\def\angle{5} % Angle of the inclined plane
			\def\desl{(8,1)}
			\resizebox{0.8\textwidth}{!}{
	% \def\angle{20} % Angle of the inclined plane
	% \def\desl{(5,1)}
			\begin{tikzpicture}
				\begin{scope}
					\node [fill=blue, draw=black, shape=circle] (0) at (-1.5, 4.5) {};
					\node [fill=blue, draw=black, shape=circle] (1) at (-0.75, 4.5) {};
					\node [fill=blue, draw=black, shape=circle] (2) at (0.5, 4.5) {};
					\node [fill=blue, draw=black, shape=circle] (3) at (1.5, 4.5) {};
					\node [fill=blue, draw=black, shape=circle] (4) at (2.5, 4.5) {};
					\node [fill=blue, draw=black, shape=circle] (5) at (-2, 4.75) {};
					\node [fill=blue, draw=black, shape=circle] (6) at (-2, 6) {};
					\node [fill=blue, draw=black, shape=circle] (7) at (2.25, -5.5) {};
					\node [fill=blue, draw=black, shape=circle] (8) at (2.25, -4.5) {};
					\node [fill=blue, draw=black, shape=circle] (9) at (2.75, -4) {};
					\node [fill=blue, draw=black, shape=circle] (10) at (4, -3.75) {};
					\node [fill=blue, draw=black, shape=circle] (12) at (5.5, -1.5) {};
					\node [fill=blue, draw=black, shape=circle] (13) at (5.5, -0.5) {};
					\node [fill=blue, draw=black, shape=circle] (14) at (5.5, -2.75) {};
					\node [fill=blue, draw=black, shape=circle] (15) at (5.25, -3.5) {};
				\end{scope}
				\begin{scope}[rotate=\angle, shift={\desl}]
					\node [fill=red, draw=black, shape=circle] (out1) at (-5, 9) {};
					\node [fill=red, draw=black, shape=circle] (out2) at (-3, 8) {};
					\node [fill=red, draw=black, shape=circle] (b0) at (-1.5, 4.5) {};
					\node [fill=red, draw=black, shape=circle] (b1) at (-0.75, 4.5) {};
					\node [fill=red, draw=black, shape=circle] (b2) at (0.5, 4.5) {};
					\node [fill=red, draw=black, shape=circle] (b3) at (1.5, 4.5) {};
					\node [fill=red, draw=black, shape=circle] (b4) at (2.5, 4.5) {};
					\node [fill=red, draw=black, shape=circle] (b5) at (-2, 4.75) {};
					\node [fill=red, draw=black, shape=circle] (b6) at (-2, 6) {};
					\node [fill=red, draw=black, shape=circle] (b7) at (2.25, -5.5) {};
					\node [fill=red, draw=black, shape=circle] (b8) at (2.25, -4.5) {};
					\node [fill=red, draw=black, shape=circle] (b9) at (2.75, -4) {};
					\node [fill=red, draw=black, shape=circle] (b10) at (4, -3.75) {};
					\node [fill=red, draw=black, shape=circle] (b12) at (5.5, -1.5) {};
					\node [fill=red, draw=black, shape=circle] (b13) at (5.5, -0.5) {};
					\node [fill=red, draw=black, shape=circle] (b14) at (5.5, -2.75) {};
					\node [fill=red, draw=black, shape=circle] (b15) at (5.25, -3.5) {};
				\end{scope}

				\pause

				\draw[green, dashed] (1) -- (b1){}; \draw[green, dashed] (2) -- (b2){};
				\draw[green, dashed] (3) -- (b3){}; \draw[green, dashed] (4) -- (b4){};
				\draw[green, dashed] (6) -- (b6){}; \draw[green, dashed] (13) -- (b13){};
				\draw[green, dashed] (5) -- (b5){}; \draw[green, dashed] (8) -- (b8){};
				\draw[green, dashed] (7) -- (b7){}; \draw[green, dashed] (10) -- (b10){}; 
				\draw[green, dashed] (9) -- (b9){}; \draw[green, dashed] (12) -- (b12){};
				\draw[green,dashed] (14) -- (b14){}; \draw[green,dashed] (15) -- (b15){};
				\draw[red, dashed] (6) -- (out1){}; \draw[red, dashed] (6) -- (out2){};
				
			\end{tikzpicture}}
		\end{figure}
	\end{minipage}
\end{frame}


\begin{frame}[t, allowframebreaks]
	\frametitle{References}
	\bibliography{\RMHOME/references.bib}
\end{frame}


\end{document}