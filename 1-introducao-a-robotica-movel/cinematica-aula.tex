\documentclass{beamer}
\usetheme{simple}
\usepackage[brazil]{babel}
\usepackage[utf8]{inputenc} 
\usepackage{lmodern}
\usefonttheme[onlymath]{serif}
\usepackage[scale=2]{ccicons} 
\usepackage[makeroom]{cancel}
\usepackage{copyrightbox}

\usepackage{graphicx,hyperref,url,pgfplots}
\usepackage{amsmath} 
\usepackage{array,booktabs}
\pgfplotsset{compat=1.13}  
\usepackage{pifont}
\usepackage{bibentry}
\usepackage[alf,abnt-etal-list=0,abnt-etal-cite=3]{abntex2cite}
\usepackage[normalem]{ulem}

\setbeamercovered{invisible}
\newcommand{\pausar}{ }
\newcommand{\df}[1]{\,\mathrm{d}#1}
\newcommand{\parcial}[3]{\dfrac{\partial^{#1}#2}{\partial #3^{#1}}}
\newcommand{\cpright}[2]{\copyrightbox[b]{#1}{\tiny Source: #2}}
\newcommand{\cmark}{\textcolor{green}{\ding{51}}}%
\newcommand{\xmark}{\textcolor{red}{\ding{55}}}%

\usepackage{tikz}
\usepackage{xcolor}
\usetikzlibrary{scopes}
\usepackage{verbatim}
\usetikzlibrary{patterns}

\usepackage{listings}
	\definecolor{codegreen}{rgb}{0,0.6,0}
	\definecolor{codegray}{rgb}{0.5,0.5,0.5}
	\definecolor{codepurple}{rgb}{0.58,0,0.82}
	\definecolor{backcolour}{rgb}{0.92,0.92,0.92}
	\lstset{language=Python, 
	backgroundcolor=\color{backcolour},   
	commentstyle=\color{codegreen},
	keywordstyle=\color{magenta},
	numberstyle=\tiny\color{codegray},
	stringstyle=\color{codepurple},
	basicstyle=\fontsize{8}{11}\ttfamily,
	frame=lines,
%	numbers=left,
	tabsize=2,
	morekeywords={models, lambda, forms}}

\usepackage{catchfile}
\newcommand{\getenv}[2][]{%
    \CatchFileEdef{\temp}{"|kpsewhich --var-value #2"}{\endlinechar=-1}%
    \if\relax\detokenize{#1}\relax\temp\else\let#1\temp\fi}

\getenv[\BIBREF]{RM_REFERENCES}

% --------------------------------------------------------------------------------------------

\title{Cinemática}
\subtitle{Equações de Movimento}
\date{\today}
\author[Jeferson José de Lima]{
  \textbf{Professor}: Jeferson José de Lima}
\institute[UTFPR-PB]{Departamento de Informática (DAINF)}

\begin{document}

\maketitle

\begin{frame}{Informações Úteis}
    \begin{block}{Material disponível em:}
        \href{Robótica Móvel - Wiki}{https://gitlab.com/cursoseaulas/robotica-movel/-/wikis/home}
    \end{block}
    \pausar
    % \begin{block}{Avisos Importantes}
    %     \begin{itemize}
    %         \item \textcolor{red}{Envio Moodle: Nomes das equipes e do Robô}
    %     \end{itemize}
    % \end{block}
    \pausar
    \begin{block}{Materiais e Recursos de Aula}
        \begin{itemize}
            \item Quadros / Slides
            \item Python - Jupyter Notebook (Colab)
        \end{itemize}
    \end{block}
    
\end{frame}


\begin{frame}{Modelagem}
    \framesubtitle{Introdução}
    \begin{itemize}
        \item Tipo de Modelos:
              \begin{itemize}
                  \item Modelo Cinemático;
                  \item Modelo Dinâmico;
              \end{itemize}
    \end{itemize}
    \begin{center}
        \cpright{\includegraphics[width=0.8\textwidth]{./images/mecanismos.jpg}}
        {https://edisciplinas.usp.br/pluginfile.php/3280265}
    \end{center}
\end{frame}


\begin{frame}{Modelagem Cinemática}
    \framesubtitle{Conceitos}
    \begin{itemize}
        \item A cinemática é a área da Física que estuda o movimento dos corpos.
        \item Em robótica móvel a cinemática estabelece relações entre o deslocamento (locomoção) do robô e a atuação a ele imposta.
        \item A cinemática direta estabelece modelos que estimam o deslocamento do robô dada uma atuação, por exemplo, velocidade imposta às suas rodas. A cinemática direta está relacionada as \textbf{Coordenadas Generalizadas}.
        \item A cinemática inversa estabelece modelos que estimam a atuação necessária para que o robô realize um determinado deslocamento, por exemplo, percorrer uma trajetória \footnote{http://143.106.148.168:9080/Cursos/IA368N/01-16/cinematica2.pdf}.
    \end{itemize}
\end{frame}


\begin{frame}{Modelagem Cinemática}
    \framesubtitle{Introdução}
    Modelo Cinemático:
            \begin{itemize}
                \item Cinemático Direta e Inversa;
                \item Holonomicidade\footnote{O termo holonômico é atribuido a Hertz (Arnol'd and (Eds.), 1994) e significa "universal", "integral", "integrável" ( literalmente: holo = o todo, conjunto, totalidade - nomia = lei). Portanto, sistemas não-holonômicos podem ser interpretados como sistemas não integráveis. \textcolor{blue}{Definem-se como não-holonômicos sistemas com dimensão finita onde algum tipo de restrição é imposta a um ou mais estados do sistema.}}
            \end{itemize}

\end{frame}


\begin{frame}{Cinemática de Robôs}
    \framesubtitle{Conceitos}
    \textbf{Um robô é modelado como um corpo rígido}
        \begin{itemize}
            \item 3 variáveis $x, y,\phi$ (plano)
            \item 6 variáveis $x,y,z, \alpha, \beta, \phi$ (espaço)
        \end{itemize}
    Deve-se estabelecer uma relação entre o sistema de referencia local (robô) e o sistema de referencia global
        \begin{itemize}
            \item Sistema de Referencia global, exemplo $\{W\}$, ou $\{A\}$\footnote{O sistema de referência $\{A\}$(global) e $\{B\}$(local) serão usados como exemplos genéricos em aula};
            \item \item Referencial local, exemplo $\{M\}$, ou $\{B\}$;
        \end{itemize}
        \begin{center}
            \cpright{\includegraphics[width=0.5\textwidth]{./images/tf_1.png}}
            {https://www.kidscoding8.com/99205.html}
        \end{center}
\end{frame}



\begin{frame}{Modelagem Cinemática}
    \framesubtitle{Transformação homogênea - Conceitos}
    \begin{itemize}
        \item Um vetor de posição que pertence ao espçaço $\mathbb{R}^{3 \times 1}$ é composto pelas coordenadas $X,Y$ e $Z$.
        \item Um ponto ${}^A\mathbf{P}$ representa a distância do vetor do plano $\{A\}$. Os elementos individuais de ${}^A\mathbf{P}$ podem ser visto pela equação \eqref{eq:cine1}.
    \end{itemize}
    \begin{columns}[c]
        \begin{column}{0.6\textwidth}
            \begin{figure}
                \centering
                \begin{tikzpicture}[scale=0.8]
                    \node(p0) at (0,0){};
                    \draw [->, blue] (p0.center) --++(0,3) node[right] {$ Y_A$};
                    \draw [->, rotate =120, blue] (p0.center) --++(0,3) node[below] {$ Z_A$};
                    \draw [->, rotate =240, blue] (p0.center) --++(0,3) node[below] {$ X_A$};
                    \draw [->, black] (p0.center) --++(2.5,0.5) node(B)[above,rotate=30] {${}^A\mathbf{P}$};
                    \node at (-1.5,2.5)[, blue] {$\{A\}$};
                \end{tikzpicture}
                \caption{Vetor em relação ao sistema de referências $\{A\}$}
                \label{fig:cine1f}
            \end{figure}
        \end{column}
        \begin{column}{0.3\textwidth}
            \begin{equation}\label{eq:cine1}
                \color{black}{{}^A\mathbf{P} = \begin{bmatrix}
                    p_x \\ p_y \\ p_z
                \end{bmatrix}}
                \color{gray}{}
            \end{equation}
        \end{column}
    \end{columns}
\end{frame}



\begin{frame}{Modelagem Cinemática}
    \framesubtitle{Transformação homogênea - Matriz de Rotação}
    \begin{itemize}
        \item O vetor definido por ${}^A\mathbf{P}$ pode ser rotacionado pela matriz de rotação $\mathbf{R}$, conforme a equação \eqref{eq:cine2}.
              \begin{equation}\label{eq:cine2}
                  {}_A^B
                  \mathbf{R} =
                  \begin{bmatrix}
                      r_{11} & r_{11} & r_{11} \\
                      r_{21} & r_{21} & r_{21} \\
                      r_{31} & r_{31} & r_{31} \\
                  \end{bmatrix}
              \end{equation}
    \end{itemize}

    \begin{block}{Considerando os exemplos}
        \begin{equation*}
            \mathbf{R}(\theta) =
            \begin{bmatrix}
                \cos \theta & -\sin \theta \\\sin \theta &\cos \theta
            \end{bmatrix}
        \end{equation*}
        ou:
        \begin{equation*}
            \mathbf{R}_z(\theta) =
            \begin{bmatrix}
                \cos(\theta) & -\sin(\theta) & 0 \\
                \sin(\theta) & \cos(\theta) & 0 \\
                0            & 0            & 1 \\
            \end{bmatrix} \text{, eixo $Z$ fixo}
        \end{equation*}
    \end{block}

\end{frame}



\begin{frame}{Modelagem Cinemática}
    \framesubtitle{Transformação homogênea - Matriz de Rotação}
    \begin{itemize}
        \item Demais eixos:
    \end{itemize}
    \begin{block}{}
        \begin{equation*}
            \mathbf{R}_x(\theta) =
            \begin{bmatrix}
                1 & 0            & 0             \\
                0 & \cos(\theta) & -\sin(\theta) \\
                0 & \sin(\theta) & \cos(\theta)  \\
            \end{bmatrix} \text{, eixo $x$ fixo}
        \end{equation*}
        \begin{equation*}
            \mathbf{R}_y(\theta) =
            \begin{bmatrix}
                \cos(\theta)  & 0 & \sin(\theta) \\
                0             & 1 & 0            \\
                -\sin(\theta) & 0 & \cos(\theta) \\
            \end{bmatrix} \text{, eixo $y$ fixo}
        \end{equation*}
        \begin{equation*}
            \mathbf{R}_z(\theta) =
            \begin{bmatrix}
                \cos(\theta) & -\sin(\theta) & 0 \\
                \sin(\theta) & \cos(\theta) & 0 \\
                0            & 0            & 1 \\
            \end{bmatrix} \text{, eixo $z$ fixo}
        \end{equation*}
    \end{block}
\end{frame}

\begin{frame}{Modelagem Cinemática}
    \framesubtitle{Transformação homogênea - Matriz de Rotação - Exercício}
    \begin{block}{Exercício - Colab}
        \href{https://cursoseaulas.gitlab.io/robotica-movel/exercicio01.ipynb}{Rotação Translação e Transformação Homogênea}
    \end{block}
\end{frame}

\begin{frame}[t]{Modelagem Cinemática}
    \framesubtitle{Transformação homogênea - Rotação de um coordenada}
    \begin{itemize}
        \item A rotação do sistema de referência $\{B\}$ em torno e $Z$, de um angulo qualquer $\theta$ em ${}^A\mathbf{P}$ é descrita como na equação \eqref{eq:cine3}, onde.
              \begin{equation}\label{eq:cine3}
                  {}^A\mathbf{P} = {}_B^A \mathbf{R}(\theta) {}^B\mathbf{P} =
                  \begin{bmatrix}
                      \cos(\theta) & -\sin(\theta) & 0 & 0 \\
                      \sin(\theta) & \cos(\theta) & 0 & 0 \\
                      0            & 0            & 1 & 0 \\
                      0            & 0            & 0 & 1 \\
                  \end{bmatrix}.
                  \begin{bmatrix}
                      {}^Ap_x \\
                      {}^Ap_y \\
                      {}^Ap_z \\
                      1
                  \end{bmatrix}
              \end{equation}
              \begin{columns}
                \begin{column}[c]{0.5\textwidth}
                \begin{figure}[!ht]
                    \centering
                    \begin{tikzpicture}[scale=0.5]
                        \node(p0) at (0,0){};
                        \draw [->] (p0.center) --++(0,3) node[right] {$\hat Y_A$};
                        \draw [->, rotate =120] (p0.center) --++(0,3) node[below] {$\hat Z_A$};
                        \draw [->, rotate =240] (p0.center) --++(0,3) node[below] {$\hat X_A$};
                        \node(p1) at (6,1){};
                        \draw [->, rotate =30, red] (p0.center) --++(0,3) node[right,rotate=30] {$\hat Y_B$};
                        \draw [->, rotate =150, red] (p0.center) --++(0,3) node[below,rotate=30] {$\hat Z_B$};
                        \draw [->, rotate =270, red] (p0.center) --++(0,3) node[below,rotate=30] {$\hat X_B$};
                        \draw [->, rotate =-20, red] (p0.center) --++(1.5,2) node(B)[above,rotate=30] {${}^B\mathbf{P}$};
                    \end{tikzpicture}
                \end{figure}
            \end{column}
            \begin{column}[c]{0.5\textwidth}
                A posição de ${}^A\mathbf{P}$ (sistema de referência global) em relação a ${}^B\mathbf{P}$ é encontrado através da multiplicação da matriz de ${}_B^A \mathbf{R}(\theta)$ (lê-se rotação do sistema de referência $B$ em $A$) pela posição de ${}^B\mathbf{P}$
            \end{column}
        \end{columns}
    \end{itemize}
\end{frame}

\begin{frame}[t]{Modelagem Cinemática}
    \framesubtitle{Transformação homogênea - Translação}
    \begin{itemize}
        \item Operador de translação $\mathbf{D}(q)$:
        \begin{columns}
            \begin{column}[c]{0.7\textwidth}
              \begin{figure}[!ht]
                  \centering
                  \begin{tikzpicture}[scale=0.6]
                      \node(p0) at (0,0){};
                      \draw [->] (p0.center) --++(0,3) node[right] {$\hat Y_A$};
                      \draw [->, rotate =120] (p0.center) --++(0,3) node[below] {$\hat Z_A$};
                      \draw [->, rotate =240] (p0.center) --++(0,3) node[below] {$\hat X_A$};
                      \node(p1) at (6,0.8){};
                      \draw [->, red] (p1.center) --++(0,3) node[right] {$\hat Y_B$};
                      \draw [->, rotate =120, red] (p1.center) --++(0,3) node[below] {$\hat Z_B$};
                      \draw [->, rotate =240, red] (p1.center) --++(0,3) node[below] {$\hat X_B$};
                      \draw [->, red] (p1.center) --++(2,2) node(B){};
                      \node[above] at (B.center) {${}^B\mathbf{P}_{p_x, p_y, p_z}$};
                      \draw [dotted,-latex] (p0)  -- (p1) node[midway, fill=white]{${}^A\mathbf{D}(q)$};
                      \draw [-latex,dashed] (p0)  -- (B.center) node[midway, fill=white]{${}^A\mathbf{P}$};
                      \node at (-1.5,2.5) {$\{A\}$};
                      \node at (4,2.5)[red] {$\{B\}$};
                  \end{tikzpicture}
              \end{figure}
            \end{column}
            \begin{column}[c]{0.3\textwidth}
                A operação de translação $\mathbf{D}(q)$ desloca a origem de ${}^B\mathbf{P}$ a partir do sistema referência $\{A\}$.
            \end{column}
        \end{columns}
    \end{itemize}

    \begin{equation}
        {}^A\mathbf{P} = {}^A\mathbf{D}(q) {}^B\mathbf{P} =
        \begin{bmatrix}
            1 & 0 & 0 & q_x \\
            0 & 1 & 0 & q_y \\
            0 & 0 & 1 & q_z \\
            0 & 0 & 0 & 1 \\
        \end{bmatrix}.
        \begin{bmatrix}
            p_x \\
            p_y \\
            p_z \\
            1
        \end{bmatrix}
    \end{equation}

\end{frame}


\begin{frame}{Modelagem Cinemática}
    \framesubtitle{Transformação homogênea - Translação}
    \begin{itemize}
        \item O Deslocamento é chamado de translação, e dá-se pelo operador translacional $\mathbf{D}_A(q)$, onde ${}^A\mathbf{Q}$ o incremento da posição em relação ao sistema de referencia $\{A\}$, conform equação \eqref{eq:cine4}.
              \begin{equation}\label{eq:cine4}
                  {}^A\mathbf{Q} =
                  \begin{bmatrix}
                      q_x \\ q_y \\ q_z
                  \end{bmatrix}, \qquad \mathrm{e} \qquad
                  \mathbf{D}_A =
                  \begin{bmatrix}
                      1 & 0 & 0 & q_x \\
                      0 & 1 & 0 & q_y \\
                      0 & 0 & 1 & q_z \\
                      0 & 0 & 0 & 1
                  \end{bmatrix}.
              \end{equation}
        \item Adota-se agora a notação para translação e rotação de um vetor, conforme a equação \eqref{eq:cine5}. Observa-se que o vetor ${}^A\mathbf{Q}$ foi incorporada pela nova notação.
              \begin{equation}\label{eq:cine5}
                  \begin{bmatrix}
                      {}^A\mathbf{P} \\ 1
                  \end{bmatrix}
                  =
                  \underbrace {
                      \left[
                          \begin{matrix}
                                & {}_B^A\mathbf{R} &   \\ \hline
                              0 & 0                & 0 \\
                          \end{matrix} \right.
                          \left.
                          \vline
                          \begin{matrix}
                              {}^A\mathbf{Q} \\ \hline
                              1
                          \end{matrix} \right]
                  }_{{}^A_B\mathcal{A}}
                  \begin{bmatrix}
                      {}^B\mathbf{P} \\
                      1
                  \end{bmatrix}
              \end{equation}
    \end{itemize}
\end{frame}


\begin{frame}{Modelagem Cinemática}
    \framesubtitle{Transformação homogênea - Operadores}
    \begin{itemize}
        \item Aplicando se uma transformação nas coordenada ${}^B\mathbf{P}$ pelos operadores de rotação e translação temos a representação de  ${}^A\mathbf{P}$
              \begin{figure}[!ht]
                  \centering
                  \begin{tikzpicture}[scale=0.7]
                      \node(p0) at (0,0){};
                      \draw [->] (p0.center) --++(0,3) node[right] {$\hat Y_A$};
                      \draw [->, rotate =120] (p0.center) --++(0,3) node[below] {$\hat Z_A$};
                      \draw [->, rotate =240] (p0.center) --++(0,3) node[below] {$\hat X_A$};
                      \node(p1) at (6,1){};
                      \draw [->, rotate =30, red] (p1.center) --++(0,3) node[right,rotate=30] {$\hat Y_B$};
                      \draw [->, rotate =150, red] (p1.center) --++(0,3) node[below,rotate=30] {$\hat Z_B$};
                      \draw [->, rotate =270, red] (p1.center) --++(0,3) node[below,rotate=30] {$\hat X_B$};
                      \draw [->, rotate =30, red] (p1.center) --++(1.5,4) node(B)[above,rotate=30] {${}^B\mathbf{P}$};
                      \draw [dotted,-latex] (p0)  -- (p1) node[midway, fill=white]{$\mathbf{P}_{BORG}$\footnote{A origem de sistema de referencia $\{B\}$ foi deslocada, conforme ${}^A\mathbf{Q}$}};
                      \draw [-latex,dashed] (p0)  -- (B) node[midway, fill=white]{${}^A\mathbf{P}$};;
                      \node at (-1.5,2.5) {$\{A\}$};
                      \node at (4,2.5)  [rotate=30, red]   {$\{B\}$};
                  \end{tikzpicture}
                  \label{fig:cine2}
              \end{figure}
    \end{itemize}
\end{frame}

\begin{frame}{Modelagem Cinemática}
    \framesubtitle{Transformação homogênea - Transformação Homogênea}
    \begin{itemize}
        \item Na forma generalizada, a transformação homogênea final ${}^{i}_0\mathbf{T}$ pode ser expressa pelo produto das sucessivas transformações de ${}^{i-1}_0\mathcal{A}_i$. Conforme é mostrado na equação \eqref{fig:cine3}.
              \begin{equation}\label{fig:cine3}
                  \begin{array}{lcl}
                      {}^i_0\mathbf{T} & = & {}^0_1\mathcal{A}{}^1_2\mathcal{A} \cdots {}^{i-1}_i\mathcal{A} = \prod \limits^i_{j=1}{}^{j-1}_i\mathcal{A}, \quad \mathrm{para\;}i=1,2,\cdots,n \\[.2cm]
                                       & = &
                      \begin{bmatrix}
                          x_i & y_i & z_i & p_i \\
                          0   & 0   & 0   & 1
                      \end{bmatrix} =
                      \begin{bmatrix}
                          {}^i_0\mathbf{R} & {}^i_0\mathbf{P} \\
                          \mathbf{0}       & 1
                      \end{bmatrix}
                  \end{array}
              \end{equation}
        \item onde, ${}^i_0\mathbf{P}$ é o vetor de orientação do referencial $i$ em relação a base $0$.
    \end{itemize}

\end{frame}


\begin{frame}{Modelagem Cinemática}
    \framesubtitle{Transformação homogênea - Exemplo}
    \begin{enumerate}
        \item Considerando que $\mathbf{v}_{t-1}$ é um vetor unitário em $\mathbf{v}_{t-1}=\{1,0,0\}$ e sofre um deslocamento de $\mathbf{Q}=\{2,1,0\}$ e rotação $\mathbf{R_z(\phi)}=20^o$, qual será a posição final de $v$ no plano $\{A\}$?
    \end{enumerate}
    \begin{columns}
        \begin{column}[c]{0.5\textwidth}
            \input{./images/car_1_image.tex}
        \end{column}
        \begin{column}[c]{0.5\textwidth}
            \input{./images/car_2_image.tex}
        \end{column}
    \end{columns}
    Transformação Homogênea:
    \begin{equation*}
        \begin{bmatrix}
            \color{purple}{{}^A \mathbf{v}_{t}} \\ 1
        \end{bmatrix}
        =
        \left[
            \begin{matrix}
                  & \color{green}{{}_B^A\mathbf{R}_z(\phi)} &   \\ \hline
                0 & 0                                       & 0 \\
            \end{matrix} \right.
            \left.
            \vline
            \begin{matrix}
                \color{red}{{}^A\mathbf{Q}} \\ \hline
                1
            \end{matrix} \right]
        \begin{bmatrix}
            {}^B \mathbf{v}_{t-1} \\
            1
        \end{bmatrix}
    \end{equation*}
\end{frame}

\begin{frame}[t]{Modelagem Cinemática}
    \framesubtitle{Transformação homogênea - Resumo}
    \begin{itemize}
        \item \textcolor{purple}{\textbf{Resumo:}}
    \end{itemize}
    \begin{figure}
        \centering
        \cpright{\includegraphics[width=0.6\textwidth]{./images/tf_2.png}}
        {https://answers.ros.org/questions/265846/revisions/}
        \caption{Exemplo de sistemas de referência - Carro}
    \end{figure}
\end{frame}

\begin{frame}[t]{Modelagem Cinemática}
    \framesubtitle{Transformação homogênea - Resumo}
    \begin{itemize}
        \item \textcolor{purple}{\textbf{Resumo:}}
    \end{itemize}
    \begin{figure}
        \centering
        \cpright{\includegraphics[width=0.6\textwidth]{./images/tf_3.png}}
        {http://www.osrobotics.org/osr/planning/introduction.html}
        \caption{Exemplo de sistemas de referência - Manipulador Robótico}
    \end{figure}
\end{frame}


\begin{frame}{Cinemática Direta e Inversa}
    \framesubtitle{Coordenadas Generalizadas}
    \begin{itemize}
        \item Considerando um robô fixo ou móvel com coodenadas generalizadas $q_1, q_2,..., q_n$ localizadas no espaço das \textcolor{red}{juntas ou atuadores (\textit{joint space}) $\mathbf{q}$}. Bem como $x_1, x_2,..., x_n$, o \textcolor{blue}{espaço das tarefas (\textit{task space}) $\mathbf{x}$}, temos então os vetores:
             \begin{equation*}
                \color{red}{
                  \mathbf{q} =
                  \begin{bmatrix}
                      q_1 \\q_2 \\ \cdots  \\ q_n
                  \end{bmatrix}
                }
                  \text{, }
                \color{blue}{
                  \mathbf{x} =
                  \begin{bmatrix}
                      x_1 \\x_2 \\ \cdots \\ x_n
                  \end{bmatrix}
                }
              \end{equation*}

        \item \textbf{Cinemática Direta e Inversa}. A \textcolor{red}{Cinemática Direta} descreve o estado do robô em função de entradas como (velocidade das rodas, movimento das juntas, direção das rodas ...).  A partir da \textcolor{blue}{Cinemática Inversa}, é possível projetar um planejamento de movimento, o que significa que as entradas do robô podem ser calculadas para uma sequência de estado do robô desejada.

    \end{itemize}
\end{frame}



\begin{frame}{Cinemática Direta e Inversa}
    \framesubtitle{Coordenadas Generalizadas}
    \begin{itemize}
        \item A relação entre as Cinemática Direta e Cinemática Inversa é obtida através da Matriz Jacobiana do Robô.

              \begin{equation*}
                  \mathbf{\dot{x}} = \mathbb{J}{\mathbf{\dot{q}}}
                  \text{ e, }
                  \mathbf{\dot{q}} = \mathbb{J}^{-1}{\mathbf{\dot{x}}}
              \end{equation*}

              bem como:

              \begin{equation*}
                  \frac{\text{d}\mathbf{x}}{\text{d}t} = \mathbb{J}\frac{\text{d}\mathbf{q}}{\text{d}t}
                  \text{ e, }
                  \frac{\text{d}\mathbf{q}}{\text{d}t} = \mathbb{J}^{-1}\frac{\text{d}\mathbf{x}}{\text{d}t}
              \end{equation*}

              onde $\mathbb{J}$ é dado por:
              \begin{equation*}
                  \mathbb{J}
                  =
                  \frac{d \mathbf{f}}{d \mathbf{q}}
                  =
                  \left[ \frac{\partial \mathbf{f}}{\partial q_1}
                      \cdots \frac{\partial \mathbf{f}}{\partial q_n} \right]
                  =
                  \begin{bmatrix}
                      \frac{\partial f_1}{\partial q_1} & \cdots &
                      \frac{\partial f_1}{\partial q_n}                   \\
                      \vdots                            & \ddots & \vdots \\
                      \frac{\partial f_m}{\partial q_1} & \cdots &
                      \frac{\partial f_m}{\partial q_n}
                  \end{bmatrix}
              \end{equation*}
    \end{itemize}
\end{frame}


\begin{frame}{Cinemática de Robôs}
    \framesubtitle{Conceitos}
    \begin{itemize}
        \item O modelo cinemático não leva em conta a inércia do robô, deformações em
              sua estrutura, forças oriundas do deslocamento (atrito, escorregamento, etc.),
              e demais fatores internos e externos que possam afetar a locomoção.
        \item Os modelos dinâmicos são capazes de incorporar estas variáveis, mas são
              muito mais complexos que os modelos cinemáticos.
        \item Os modelos cinemáticos são suficientes quando a locamoção se dá a baixas
              velocidades e em piso plano e horizontal que propicie contato adequado para
              não haver escorregamento.
        \item Apesar do modelo cinemático ser inerentemente um modelo aproximado,
              podemos corrigir seus resultados a partir dos sensores do robô. Os algoritmos
              de localização robótica fazem exatamente isto.\href{http://143.106.148.168:9080/Cursos/IA368N/01-16/cinematica2.pdf}{[1]}
    \end{itemize}
\end{frame}

\begin{frame}{Cinemática de Robôs}
    \framesubtitle{Diversos modelos de cinemática}
    \begin{itemize}
        \item \textbf{Cinemática Externa} descreve a posição do robô e orientação com relação ao sistema de referência externo, como por exemplo a relacao entre o robô e as coordenadas de um mapa global.
        \item \textbf{Cinemática Interna} Há a possibilidade também, da referência ser o próprio robô ou em relação as rodas.
        \item \textbf{Restrições de movimento} aparecem quando um sistema tem menos váriáveis de entrada do que graus de liberdade (DOF's).
    \end{itemize}
\end{frame}


\begin{frame}{Cinemática de Robôs}
    \framesubtitle{Cinemática Externa}
    \begin{itemize}
        \item O deslocamento de um robô deve ser expresso em relação a um sistema de
              coordenadas (referencial) inercial (global). No plano, utilizamos coordenadas
              cartesianas (eixos X e Y).
    \end{itemize}

    \begin{columns}
        \begin{column}[c]{0.5\textwidth}
            \input{./images/car_1a_image.tex}
        \end{column}
        \begin{column}[c]{0.5\textwidth}
            \input{./images/car_2a_image.tex}
        \end{column}
    \end{columns}

    \begin{equation*}
        \begin{bmatrix}
            \color{purple}{x_I} \\ \color{purple}{y_I} \\ \color{purple}{z_I}\\1
        \end{bmatrix}
        =
        \begin{bmatrix}
            \color{purple}{{}^M_I\mathbf{P}} \\ 1
        \end{bmatrix}
        =
        \left[
            \begin{matrix}
                  & \color{green}{{}_M^I\mathbf{R}_z(\phi)} &   \\ \hline
                0 & 0                                       & 0 \\
            \end{matrix} \right.
            \left.
            \vline
            \begin{matrix}
                \color{red}{{}^I\mathbf{Q}} \\ \hline
                1
            \end{matrix} \right]
        \begin{bmatrix}
            {}^M\mathbf{P} \\
            1
        \end{bmatrix}
    \end{equation*}

\end{frame}


\begin{frame}{Cinemática de Robôs}
    \framesubtitle{Cinemática Interna}
    \begin{columns}
        \begin{column}[c]{0.5\textwidth}
            \centering
            \def\iangle{35} % Angle of the inclined plane
\def\down{0}
\def\arcr{0.7cm} % Radius of the arc used to indicate angles
\newcommand\centerofmass{%
    \tikz[radius=0.2em] {%
        \fill (0,0) -- ++(0.2em,0) arc [start angle=0,end angle=90] -- ++(0,-0.4em) arc [start angle=270, end angle=180];%
        \draw (0,0) circle;%
    }%
}

\begin{tikzpicture}[
    force/.style={>=latex,draw=blue,fill=blue},
    axis/.style={densely dashed,gray,font=\small},
    M/.style={rectangle,draw,fill=lightgray,minimum size=0.7cm,thin},
    m/.style={rectangle,draw=black,fill=lightgray,minimum size=0.3cm,thin},
    plane/.style={draw=black,fill=blue!10},
    string/.style={draw=red, thick},
    pulley/.style={thick},
    wheel/.style={fill=black, rounded corners=1.5pt},
]
    %% Free body diagram of M
    \begin{scope}[rotate=\iangle]
        \node[M,transform shape] (M) {\centerofmass};
        % Draw axes and help lines
        {[axis,->]
            \draw (M) -- ++(0,1.3) node(y1_axis)[right] {$y$};
            \draw (M) -- ++(2,0) node[right] {$x$};
            % Indicate angle. The code is a bit awkward.
            \draw[solid,shorten >=0.5pt] (\down-\iangle:\arcr)
                arc(\down-\iangle:\down:\arcr);
            \node at (\down-0.5*\iangle:1.3*\arcr) {$\phi$};
        }
        % Forces
        {[force,->]
            % Assuming that Mg = 1. The normal force will therefore be cos(alpha)
            \draw (M.east) -- ++(1,0) node[above, blue] {$v_M$};
        }
        \draw[wheel] (M.south west) rectangle ++(.4,-.1) node[below]{$v_{M_R}$};
        \draw[wheel] (M.north west) rectangle ++(.4,.1)  node[left]{$v_{M_L}$};
        \draw [-](M) -- ++(0,2) node(CIR)[above] {CIR};
    \end{scope}
    % Draw gravity force. The code is put outside the rotated
    % scope for simplicity. No need to do any angle calculations. 
    \draw[axis,] (M.center) -- ++(1,0) node[below] {};
    %%
    \node[right, gray,font=\small, xshift=8] at (y1_axis) {$\{M\}$};
    %%
    \draw[, ->] (-2,-1) -- ++(4,0) node[below] {$X$};
    \draw[, ->] (-2,-1) -- ++(0,3) node(y_axis)[right] {$Y$};
    \draw[gray, ->] (-2,-1) -- ++(-.5,-.5) node[left] {$Z$};
    \node[left, gray,font=\small, xshift=-10] at (y_axis) {$\{I\}$};
\end{tikzpicture}

        CIR - Centro Instantâneo de Rotação
        \end{column}
        \begin{column}[c]{0.5\textwidth}
            \centering
            \begin{itemize}
                \item Posição:
                      \newline

                      $\mathbf{x} = \begin{bmatrix}
                              x \\
                              y
                          \end{bmatrix}$
                      \newline

                \item Configuração (localização e orientação):
                      \newline

                      $\mathbf{q} =
                          \begin{bmatrix}
                              x \\
                              y \\
                              \phi
                          \end{bmatrix}$
            \end{itemize}
        \end{column}
    \end{columns}
\end{frame}


\begin{frame}{Cinemática de Robôs}
    \framesubtitle{Robô Diferencial}
    \begin{itemize}
        \item Analisando a velocidade angular de $\omega$, temos:

              \begin{equation*}
                  \begin{split}
                      \omega & = \frac{v_L(t)}{R(t)-\frac{L}{2}} \\
                      \omega & = \frac{v_R(t)}{R(t)+\frac{L}{2}} \\
                  \end{split}
              \end{equation*}

              logo:

              \begin{equation*}
                  \begin{split}
                      \omega (t) = \frac{ v_R(t) - v_L(t)}{L} & \text{, e }
                      R(t)  = \frac{L v_R(t) + v_L(t)}{2 v_R(t) - v_L(t)} \\
                  \end{split}
              \end{equation*}

              assim, a velocidade tangencial do veiculo é dada por:

              \begin{equation}
                  v (t) =\omega (t) R(t) = \frac{ v_R(t) - v_L(t)}{2}
              \end{equation}
    \end{itemize}
\end{frame}



\begin{frame}{Cinemática de Robôs}
    \framesubtitle{Robô Diferencial}
    \begin{itemize}
        \item As velocidades tangenciais $v_L(t)=r\omega_L(t)$ e $v_R(t)=r\omega_R(t)$, temos então a cinemática interna do robô (\textcolor{red}{coordenadas locais}):

              \begin{equation*}
                  \boxed{
                      \begin{bmatrix}
                          \dot{x}_M(t) \\
                          \dot{y}_M(t) \\
                          \dot{\phi}(t)
                      \end{bmatrix}
                      =
                      \begin{bmatrix}
                          \dot{v}_M(t) \\
                          \dot{v}_M(t) \\
                          \dot{\omega}(t)
                      \end{bmatrix}
                      =
                      \begin{bmatrix}
                          \frac{r}{2}  & \frac{r}{2} \\
                          0            & 0           \\
                          -\frac{r}{L} & \frac{r}{L}
                      \end{bmatrix}
                      \begin{bmatrix}
                          \omega_L(t) \\
                          \omega_R(t)
                      \end{bmatrix}}
              \end{equation*}

        \item e em \textcolor{red}{coordenadas globais}
              \footnote{adicionar o modelo em funcao das velocidades das rodas}:

              \begin{equation*}
                  \boxed{
                      \begin{bmatrix}
                          \dot{x}(t) \\
                          \dot{y}(t) \\
                          \dot{\phi}(t)
                      \end{bmatrix}
                      =
                      \begin{bmatrix}
                          \cos(\phi(t)) & 0 \\
                          \sin(\phi(t)) & 0 \\
                          0             & 1
                      \end{bmatrix}
                      \begin{bmatrix}
                          v(t) \\
                          \omega(t)
                      \end{bmatrix}}
              \end{equation*}

              onde as variáveis $v(t)$ e $\omega(t)$


    \end{itemize}
\end{frame}


\begin{frame}{Cinemática de Robôs}
    \framesubtitle{Robô Diferencial}
    \begin{itemize}
        \item Mas o \textit{encoder} fornece apenas a posição das rodas, como calcular a velocidade?

              ... \textbf{Aproximação de Euler}, Aproximação de Tustin, Transformação Bilinear...

              \begin{equation*}
                  \begin{split}
                      x_{k+1} &= x_k + v_k T_s\cos(\phi_k) \\
                      y_{k+1} &= y_k + v_k T_s\sin(\phi_k) \\
                      \phi_{k+1} &= \phi_k + \omega_k T_s \\
                  \end{split}
              \end{equation*}
    \end{itemize}
\end{frame}


\begin{frame}{Cinemática de Robôs}
    \framesubtitle{Robô Diferencial}

    continuar ...

\end{frame}


  
\begin{frame}[t, allowframebreaks]
	\frametitle{Referências}
	\bibliography{\BIBREF}
\end{frame}
  


\end{document}