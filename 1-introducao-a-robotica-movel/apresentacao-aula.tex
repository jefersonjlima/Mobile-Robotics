\documentclass{beamer}
\usetheme{simple}
\usepackage[brazil]{babel}
\usepackage[utf8]{inputenc} 
\usepackage{lmodern}
\usefonttheme[onlymath]{serif}
\usepackage[scale=2]{ccicons}
 
\usepackage{graphicx,hyperref,url,pgfplots}
\usepackage{amsmath} 
\usepackage{array,booktabs}
\pgfplotsset{compat=1.15} 

\setbeamercovered{invisible}
\newcommand{\pausar}{\pause}
\newcommand{\df}[1]{\,\mathrm{d}#1}
\newcommand{\parcial}[3]{\dfrac{\partial^{#1}#2}{\partial #3^{#1}}}

\usepackage{tikz}
\usepackage{xcolor}
\usetikzlibrary{scopes}
\usepackage{verbatim}
\usetikzlibrary{patterns}

\usepackage{listings}
	\definecolor{codegreen}{rgb}{0,0.6,0}
	\definecolor{codegray}{rgb}{0.5,0.5,0.5}
	\definecolor{codepurple}{rgb}{0.58,0,0.82}
	\definecolor{backcolour}{rgb}{0.92,0.92,0.92}
	\lstset{language=Python, 
	backgroundcolor=\color{backcolour},   
	commentstyle=\color{codegreen},
	keywordstyle=\color{magenta},
	numberstyle=\tiny\color{codegray},
	stringstyle=\color{codepurple},
	basicstyle=\fontsize{8}{11}\ttfamily,
	frame=lines,
%	numbers=left,
	tabsize=2,
	morekeywords={models, lambda, forms}}



% --------------------------------------------------------------------------------------------

\title{Robótica Móvel}
% \subtitle{Visão Geral}
\date{\today}
\author[Jeferson José de Lima]{
  \textbf{Professor}: Jeferson José de Lima}
\institute[UTFPR-PB]{Departamento de Informática (DAINF)}

\begin{document}

\maketitle

\begin{frame}{Informações Úteis}
	\begin{block}{Material disponível em:}
		\begin{enumerate}
			\item \href{Robótica Móvel - Wiki}{https://gitlab.com/cursoseaulas/robotica-movel/-/wikis/home}
			\item Moodle - Robótica Móvel
		\end{enumerate}
	\end{block}
	\begin{block}{Dinâmica de Aula}
		\begin{enumerate}
			\item \textbf{Aulas Teóricas}: \textcolor{red}{Segunda-feira}
			\item \textbf{Aulas Práticas}: \textcolor{blue}{Sexta-feira}
		\end{enumerate}
	\end{block}
	\begin{block}{Requisitos da Disciplina}
		\begin{itemize}
		\item Teoria de Controle
		\item Eletrônica
		\item Linguagem de Programação - \textbf{Python} e \textbf{C++}
		\item Noções básicas de Mecânica
		\end{itemize}
	\end{block}
\end{frame}


\begin{frame}{Plano de Aula}
	\begin{block}{Ementa da Disciplina}
		\begin{enumerate}
			\item Introdução a Robótica Móvel
			\item Percepção e Ação
			\item Paradigmas de Controle
			\item Ambiente de Simulação
			\item Localização e Mapeamento
		\end{enumerate}
	\end{block}

	\begin{block}{Avaliações}
		\begin{enumerate}
			\item Desenvolvimento do Projeto Prático
				\begin{itemize}
					\item Competição Livre - FPV
					\item Controle e Telemetria
					\item Mapeamento e Localização
				\end{itemize}
			\item Relatórios
		\end{enumerate}
	\end{block}

\end{frame}



\begin{frame}{Plano de Aula}
	\begin{block}{Avaliações}
		\begin{enumerate}
			\item Desenvolvimento do Projeto Prático
				\begin{itemize}
					\item Competição Livre - FPV
					\item Controle e Telemetria
					\item Mapeamento e Localização
				\end{itemize}
			\item Relatórios
		\end{enumerate}
	\end{block}
	\begin{block}{Peso das Avaliações}
		\begin{tabular}{lll}
			\hline
			$N_1$ &= Projeto Fase 1 * 0,7&+ Exercício/Relatório * 0,3\\ 
			$N_2$ &= Projeto Fase 2 * 0,7&+ Exercício/Relatório * 0,3 \\ 
			$N_3$ &= Projeto Fase 3 * 0,8&+ Exercício/Relatório * 0,2 \\ 
			\hline
		\end{tabular}
	\end{block}
\end{frame}


\begin{frame}[t]{Referências}
    \begin{itemize}
        \item Introduction to Mobile Robot Control (contexto historico)
    \end{itemize}
\end{frame}
\end{document}


% https://ethz.ch/content/dam/ethz/special-interest/mavt/robotics-n-intelligent-systems/rsl-dam/documents/RobotDynamics2016/0-introduction.pdf
