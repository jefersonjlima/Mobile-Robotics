\documentclass[aspectratio=169]{beamer}
\usetheme{simple}
\usepackage[english]{babel}
\usepackage[utf8]{inputenc} 
\usepackage{lmodern}
\usepackage{ragged2e}
\usefonttheme[onlymath]{serif} 
\usepackage[scale=2]{ccicons} 
% \setbeamertemplate{caption}[numbered]
\usepackage{copyrightbox}

\usepackage{graphicx,hyperref,url,pgfplots}
\usepackage{amsmath} 
\usepackage{array,booktabs}
\pgfplotsset{compat=1.13}
\usepackage{bibentry}
\usepackage[alf,abnt-etal-list=0,abnt-etal-cite=2]{abntex2cite}
\usepackage[normalem]{ulem}

\usepackage[
    type={CC},
    modifier={by-nc-sa},
    version={4.0},
]{doclicense}

\setbeamercovered{invisible} 
% \newcommand{\pausar}{\pause}
\newcommand{\pausar}{\pause}
\newcommand{\df}[1]{\,\mathrm{d}#1}
\newcommand{\parcial}[3]{\dfrac{\partial^{#1}#2}{\partial #3^{#1}}}
\newcommand{\cpright}[2]{\copyrightbox[b]{#1}{\tiny Source: #2}}

\usepackage{tikz}
\usetikzlibrary{automata,positioning}
\usepackage{xcolor}
\usetikzlibrary{scopes}
\usepackage{verbatim}
\usetikzlibrary{patterns}

\usepackage{listings}
  \lstdefinestyle{ascii-tree}{
    literate={├}{|}1 {─}{--}1 {└}{+}1 
  }
	\definecolor{codegreen}{rgb}{0,0.6,0}
	\definecolor{codegray}{rgb}{0.5,0.5,0.5}
	\definecolor{codepurple}{rgb}{0.58,0,0.82}
	\definecolor{backcolour}{rgb}{0.92,0.92,0.92}
	\lstset{language=Python, 
	backgroundcolor=\color{backcolour},   
	commentstyle=\color{codegreen},
	keywordstyle=\color{magenta},
	numberstyle=\tiny\color{codegray},
	stringstyle=\color{codepurple},
	basicstyle=\fontsize{8}{11}\ttfamily,
	frame=lines,
%	numbers=left,
	tabsize=2,
	morekeywords={models, lambda, forms},
	showstringspaces=false}


\usepackage{catchfile}
\newcommand{\getenv}[2][]{%
    \CatchFileEdef{\temp}{"|kpsewhich --var-value #2"}{\endlinechar=-1}%
    \if\relax\detokenize{#1}\relax\temp\else\let#1\temp\fi}

\getenv[\BIBREF]{RM_REFERENCES}
% --------------------------------------------------------------------------------------------

\title{Mobile Robots}
\subtitle{Quaternions}
\date{\today}
\author[Jeferson José de Lima]{
  \textbf{Professor}: Jeferson José de Lima}
\institute{Academic Department of Informatics (DAINF) \\ Federal University of Technology - Paraná (UTFPR) at Pato Branco, PR, Brazil}

\begin{document}
\maketitle
\justify


\begin{frame}{Useful Information}
    \begin{block}{License}
        \doclicenseThis
    \end{block}

    \begin{block}{links:}
        \begin{enumerate}
            \item \href{https://moodle.utfpr.edu.br/course/view.php?id=14218}{Moodle}
            \item \href{https://gitlab.com/cursoseaulas/robotica-movel/-/wikis/home}{Mobile Robots - Gitlab Page}
        \end{enumerate}
    \end{block}
\end{frame}


\begin{frame}{Quaternions}
    \framesubtitle{What is a quaternion?}
    \begin{itemize}
        \item What is a quaternion?
        \item Representing rotation ...
        \item Geometric view ...
        \item Transformations to other representations!
        \item Topological and metric properties
    \end{itemize}

\end{frame}


\begin{frame}{Quaternions}
    \framesubtitle{Why can't we invert vectors in $\mathbf{R}^3$?}

    \begin{itemize}
        \item We can invert reals.  $x \times \displaystyle\frac{1}{x} = 1$.
        \item We can invert elements of $\mathbf{R}^2$ using complex numbers.
        \item $z \times  \displaystyle\frac{z^*}{|z|^2} = 1$, where $^*$ is complex conjugate.
        \item Can we invert $\mathbf{v} \in \mathbf{R}^3$? \textcolor{red}{No}
        \item How about $\mathbf{v} \in \mathbf{R}^4$? \textcolor{blue}{Yes!}
        \item Hamilton's quaternions are to $\mathbf{R}^3$.
        \item what complex numbers are to $\mathbf{R}$?.
    \end{itemize}

\end{frame}

\begin{frame}{Quaternions}
    \framesubtitle{Complex numbers versus Quaternions}

    \begin{itemize}
        \item To define complex numbers  \cite{jeffcourse_2007}:
              \begin{itemize}
                  \item Basis elements $1$ and $i$;
                  \item Vector space over reals:  elements have the form $x+i y$;
                  \item One more axiom required:  $i^2$ = -1.
              \end{itemize}
        \item To define quaternions:
              \begin{itemize}
                  \item Basis elements $1$, $i$, $j$, $k$;
                  \item Vector space over reals:  elements have the form $q_0 + q_1 i + q_2 j + q_3 k$;
                  \item Six more axioms:
                        \begin{align*}
                            i^2 & = j^2 = k^2 = -1 \\
                            ij  & = k              \\
                            jk  & = i              \\
                            ki  & = j
                        \end{align*}
              \end{itemize}
    \end{itemize}
\end{frame}

\begin{frame}{Quaternions}
    \framesubtitle{Quaternion notation}

    We can write a quaternion several ways:
    \begin{align*}
        q & = q_0 + q_1 i + q_2 j + q_3 k \\
        q & = (q_0, q_1, q_2, q_3)        \\
        q & = q_0 + \mathbf{q}
    \end{align*}
    where $q_0$ is the \emph{scalar part} and $\mathbf{q}$ is the \emph{vector part}
\end{frame}


\begin{frame}{Quaternions}
    \framesubtitle{Quaternion product}

    We can write a quaternion product several ways:
    \begin{align*}
        pq & = (p_0 + p_1 i + p_2 j + p_3 k)  (q_0 + q_1 i + q_2 j + q_3 k)             \\
           & = (p_0 q_0 - p_1 q_1 - p_2 q_2 - p_3 q_3) + \ldots i + \ldots j + \ldots k \\
        pq & = (p_0 + \mathbf{p}) (q_0 + \mathbf{q})                                    \\
           & = (p_0 q_0 + p_0 \mathbf{q} + q_0 \mathbf{p} + \mathbf{p} \mathbf{q})
    \end{align*}
    So what is $\mathbf{p} \mathbf{q}$?  Cross product?  Dot product?

    Both!  Cross product minus dot product!

    \begin{equation}
        pq = (p_0 q_0 - \mathbf{p} \cdot \mathbf{q} + p_0 \mathbf{q} + q_0 \mathbf{p} + \mathbf{p} \times \mathbf{q})
    \end{equation}

\end{frame}


\begin{frame}{Quaternions}
    \framesubtitle{Conjugate, length}

    \begin{itemize}
        \item Quaternion \textbf{conjugate}:
              \begin{equation}
                  q^* = q_0 - q_1 i - q_2 j - q_3 k
              \end{equation}
        \item Note that
              \begin{align*}
                  q q^* & = (q_0 + \mathbf{q})(q_0 - \mathbf{q})                               \\
                        & = q_0^2 + q_0 \mathbf{q} - q_0 \mathbf{q} - \mathbf{q} \mathbf{q}    \\
                        & = q_0^2 + \mathbf{q} \cdot \mathbf{q} - \mathbf{q} \times \mathbf{q} \\
                        & = q_0^2 + q_1^2 + q_2^2 + q_3^2
              \end{align*}
        \item Quaternion \textbf{length}:
              \begin{equation}
                  |q| = \sqrt{q q^*} = \sqrt{q_0^2 + q_1^2 + q_2^2 + q_3^2}
              \end{equation}
    \end{itemize}
\end{frame}


\begin{frame}{Quaternions}
    \framesubtitle{Quaternion inverse}
    \begin{itemize}
        \item[]Note that every quaternion other than the additive identity $0$ has an inverse:
              \[ q^{-1} = \frac{q^*}{|q|^2} \]
    \end{itemize}
    That means quaternions are a linear algebra and a field.  Hamilton's dream.  Quaternions are the only extension of  complex numbers that is both a linear algebra and a field.  If 1D numbers are the reals, and 2D numbers are the complex numbers, then 4D numbers are quaternions, and that's all there is.  (Frobenius?)
\end{frame}

\begin{frame}{Quaternions}
    \framesubtitle{Rotation using unit quaternions}

    \begin{itemize}
        \item Let $q$ be a unit quaternion, i.e. $|q|=1$.
              \begin{itemize}
                  \item  It can be expressed as
                        \begin{equation}
                            q = \cos \frac{\theta}{2} + \sin \frac{\theta}{2} \hat{\mathbf{n}}
                        \end{equation}
              \end{itemize}
        \item Let $x = 0 + \mathbf{x}$ be a "pure vector".
        \item Let $x^\prime = qxq^*$.
        \item Then $x^\prime$ is the pure vector $\operatorname(\theta,\hat{\mathbf{n}}) \mathbf{x}$!!!
    \end{itemize}
\end{frame}


\begin{frame}{Quaternions}
    \framesubtitle{Why $\theta / 2$?  Why $qxq^*$ instead of $qx$?}

    Two puzzling things.  In analogy with complex numbers, why not use
    \begin{align*}
        p                 & = \cos\theta + \hat{\mathbf{n}}\sin\theta \\
        \mathbf{x}^\prime & = p\mathbf{x}
    \end{align*}
    To explore that idea, define a map $L_p(q) = pq$ with $p$ a unit pure vector.
    Note that $L_p(q)$ can be written:
    \begin{equation*}
        L_p(q) = \left(\begin{array}{cccc}
            p_0 & -p_1 & -p_2 & -p_3 \\
            p_1 & p_0  & -p_3 & p_2  \\
            p_2 & p_3  & p_0  & -p_1 \\
            p_3 & -p_2 & p_1  & p_0  \\
        \end{array}\right)
        \left(\begin{array}{c}
            q_0 \\ q_1 \\ q_2 \\ q_3 \\
        \end{array}\right)
    \end{equation*}
    Note that the matrix above is orthonormal.  $L_p$ is a rotation of
    Euclidean 4 space!
    (Without even using the fact that $p$ is a pure vector.)
\end{frame}


\begin{frame}{Quaternions}
    \framesubtitle{Why $\theta / 2$?  Why $qxq^*$ instead of $qx$?}
    \emph{not} a rotation of the 3D subspace of pure vectors.  Some of
    the 3D subspace leaks into the fourth dimension.

    Consider an example using $p = i$.  Is it a rotation about $i$ of $\pi/2$?
    \begin{figure}[h!]
        \centerline{
            \includegraphics[height = 5cm]{images/quatrot.png}}
    \end{figure}
\end{frame}


\begin{frame}{Quaternions}
    \framesubtitle{What do we do with a representation?}
    Rotate a point:  $qxq^*$.

    Compose two rotations:
    \begin{equation}
        q(p\mathbf{x} p^*)q^* = (qp)\mathbf{x}(qp)^*
    \end{equation}

    Convert to other representations:
    \begin{itemize}
        \item  From axis-angle to quaternion:
              \begin{equation}
                  q = \cos \frac{\theta}{2} + \sin \frac{\theta}{2} \hat{\mathbf{n}}
              \end{equation}

        \item  From quaternion to axis-angle:
              \begin{align*}
                  \theta           & = 2 \tan^{-1}(|{\mathbf{q}}|, q_0) \\
                  \hat{\mathbf{n}} & = {\mathbf{q}}/|{\mathbf{q}}|
              \end{align*}
              assuming $\theta$ is nonzero.
    \end{itemize}
\end{frame}


\begin{frame}{Quaternions}
    \framesubtitle{From quaternion to rotation matrix}

    Just expand the product
    \begin{equation}
        qxq^* =
        \left( \begin{array}{ccc}
                q_0^2 + q_1^2 - q_2^2 - q_3^2
                 & 2(q_1 q_2 - q_0 q_3)
                 & 2(q_1 q_3 + q_0 q_2)          \\
                2(q_1 q_2 + q_0 q_3)
                 & q_0^2 - q_1^2 + q_2^2 - q_3^2
                 & 2 (q_2 q_3 - q_0 q_1)         \\
                2(q_1 q_3 - q_0 q_2)
                 & 2 (q_2 q_3 + q_0 q_1)
                 & q_0^2 - q_1^2 - q_2^2 + q_3^2 \\
            \end{array}\right) \mathbf{x}
    \end{equation}
\end{frame}



\begin{frame}{Quaternions}
    \framesubtitle{From rotation matrix to quaternion}

    \begin{itemize}
        \item Given $R = (r_{ij})$, solve expression on previous page for
              quaternion elements $q_i$
        \item Linear combinations of diagonal elements seem to solve the problem:
              \begin{align*}
                  q_0^2 = & \frac{1}{4} (1 + r_{11} + r_{22} + r_{33}) \\
                  q_1^2 = & \frac{1}{4} (1 + r_{11} - r_{22} - r_{33}) \\
                  q_2^2 = & \frac{1}{4} (1 - r_{11} + r_{22} - r_{33}) \\
                  q_3^2 = & \frac{1}{4} (1 - r_{11} - r_{22} + r_{33})
              \end{align*}

              so take four square roots and you're done?  You have to figure the
              signs out.  There is a better way $\ldots$
    \end{itemize}

\end{frame}

\begin{frame}{Quaternions}
    \framesubtitle{Look at the off-diagonal elements}
    \begin{itemize}
        \item[]
              \begin{align*}
                  q_0 q_1 = & \frac{1}{4}(r_{32} - r_{23}) \\
                  q_0 q_2 = & \frac{1}{4}(r_{13} - r_{31}) \\
                  q_0 q_3 = & \frac{1}{4}(r_{21} - r_{12}) \\
                  q_1 q_2 = & \frac{1}{4}(r_{12} + r_{21}) \\
                  q_1 q_3 = & \frac{1}{4}(r_{13} + r_{31}) \\
                  q_2 q_3 = & \frac{1}{4}(r_{23} + r_{32})
              \end{align*}

        \item[]  Given any one $q_i$, could solve the above for the other three.
    \end{itemize}
\end{frame}


% \begin{frame}{Quaternions}
%     \framesubtitle{Exercice}
%         The Euler angles can be obtained from the quaternions via the relations:[3]


%         \begin{equation}
%             \begin{bmatrix}\phi \\\theta \\\psi \end{bmatrix}}={\begin{bmatrix}{\mbox{arctan}}{\frac {2(q_{0}q_{1}+q_{2}q_{3})}{1-2(q_{1}^{2}+q_{2}^{2})}}\\{\mbox{arcsin}}(2(q_{0}q_{2}-q_{3}q_{1}))\\{\mbox{arctan}}{\frac {2(q_{0}q_{3}+q_{1}q_{2})}{1-2(q_{2}^{2}+q_{3}^{2})}}\end{bmatrix}
%         \end{equation}

%         Note, however, that the arctan and arcsin functions implemented in computer languages only produce results between −π/2 and π/2, and for three rotations between −π/2 and π/2 one does not obtain all possible orientations. To generate all the orientations one needs to replace the arctan functions in computer code by atan2:

%         \begin{equation}
%             \begin{bmatrix}\phi \\\theta \\\psi \end{bmatrix}}={\begin{bmatrix}{\mbox{atan2}}(2(q_{0}q_{1}+q_{2}q_{3}),1-2(q_{1}^{2}+q_{2}^{2}))\\{\mbox{asin}}(2(q_{0}q_{2}-q_{3}q_{1}))\\{\mbox{atan2}}(2(q_{0}q_{3}+q_{1}q_{2}),1-2(q_{2}^{2}+q_{3}^{2}))\end{bmatrix}}
%         \end{equation}
% \end{frame}


\begin{frame}{Quaternions}
    \framesubtitle{Exercice}

\end{frame}


\begin{frame}[t, allowframebreaks]
    \frametitle{Bibliography}
    \bibliography{\BIBREF}
\end{frame}

\end{document}
