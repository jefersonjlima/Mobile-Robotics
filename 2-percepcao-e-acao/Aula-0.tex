\documentclass[t]{beamer}
\usepackage[brazil]{babel}
\usepackage[utf8]{inputenc}
\usepackage{graphicx,hyperref,url,pgfplots}
\usetheme{Warsaw}
% \usecolortheme{crane}
\usefonttheme[onlymath]{serif}
% previsualização animacao
\setbeamercovered{invisible}
% texto justificado
\usepackage{ragged2e}
\justifying

\pgfplotsset{compat=1.10} 
\usepackage{amsmath} 
\usepackage{array,booktabs}

% tipos de slides
\newcommand{\pausar}{\pause}

% paginas numero
\addtobeamertemplate{navigation symbols}{}{%
    \usebeamerfont{footline}%
    \usebeamercolor[fg]{footline}%
    \hspace{1em}%
    \insertframenumber/\inserttotalframenumber
}

\usepackage{listings}
	\definecolor{codegreen}{rgb}{0,0.6,0}
	\definecolor{codegray}{rgb}{0.5,0.5,0.5}
	\definecolor{codepurple}{rgb}{0.58,0,0.82}
	\definecolor{backcolour}{rgb}{0.92,0.92,0.92}
	\lstset{language=Python, 
	backgroundcolor=\color{backcolour},   
	commentstyle=\color{codegreen},
	keywordstyle=\color{magenta},
	numberstyle=\tiny\color{codegray},
	stringstyle=\color{codepurple},
	basicstyle=\fontsize{8}{11}\ttfamily,
	frame=lines,
%	numbers=left,
	tabsize=2,
	morekeywords={models, lambda, forms}}

\logo{\includegraphics[scale=0.09]{images/logo.jpg}}
\newcommand{\nologo}{\setbeamertemplate{logo}{}} % command to set the logo to nothing

\title[\textit{Conceitos de Percepção e Ação}]{
  Percepção e Ação}

% Optional: a subtitle to be dispalyed on the title slide
\subtitle{Revisão Probabilidade}

% The author(s) of the presentation:
%  - again first a short version to be displayed at the bottom;
%  - next the full list of authors, which may include contact information;
\author[Professor: Jeferson José de Lima]{
  \textbf{Professor}: Jeferson José de Lima \\\medskip
  {\small \url{jefersonlima@utfpr.edu.br}}}

% The institute:
%  - to start the name of the university as displayed on the top of each slide
%    this can be adjusted such that you can also create a Dutch version
%  - next the institute information as displayed on the title slide
\institute[UTFPR-PB]{
  Departamento de Informática (DAINF)}

% Add a date and possibly the name of the event to the slides
%  - again first a short version to be shown at the bottom of each slide
%  - second the full date and event name for the title slide
\date[2020.1]{\textbf{Disciplina}: Robótica Móvel}

\begin{document}

\begin{frame}
  \titlepage
\end{frame}

\begin{frame}
	\frametitle{Sumário}
	{\small {\small \tableofcontents}}
\end{frame}

% Section titles are shown in at the top of the slides with the current section 
% highlighted. Note that the number of sections determines the size of the top 
% bar, and hence the university name and logo. If you do not add any sections 
% they will not be visible.

\section{Informações Úteis}
\begin{frame} 
	\frametitle{Informações Úteis}
	\begin{block}{Material disponível em:}
		\href{Robótica Móvel - Wiki}{https://gitlab.com/cursoseaulas/robotica-movel/-/wikis/home}
	\end{block}
	\pausar
	\begin{alertblock}{Datas Importantes}
		\begin{itemize}
		\item Entrega
		\item Envio
		\end{itemize}
	\end{alertblock}

\end{frame}

%-----------------------------------------------------------------
\section{Revisão - Probabilidade}
\subsection{Eventos Especiais}
\begin{frame}

União, interseção e complementação de eventos:

\begin{itemize}
	\item $A  \cup B$ é o evento que ocorre se (e somente se) pelo menos um dos eventos, $A$ ou $B$, ocorrer.
	\item $A \cap B$ é o evento que ocorre se ambos, $A$ e $B$, ocorrerem simultaneamente.
	\item $A^c$ , chamado evento complementar de $A$, é o evento cujos resultados pertencem a $\Omega$ mas não a $A$.
\end{itemize}

\begin{exampleblock}{Para}
	Um espaço amostral $\Omega$ associado a um experimento aleatório, e sejam $A$ e $B$ dois eventos
	contidos em $\Omega$. Diremos que $A$ e $B$ são mutuamente exclusivos se eles não possuem elementos comuns,
	isto é, se $A \cap B = \varnothing$.
\end{exampleblock}

\end{frame}
%-----------------------------------------------------------------
\begin{frame}

\begin{exampleblock}{Exemplo}

	Consideremos o lançamento de um dado equilibrado.
	
	Sejam $A= \{\text{número par}\} , B = \{\text{número maior que}\} \text{ e } C =\{ 3 \}$.
	
	Então $A =\{ 2,4,6 \} , B = \{ 5,6 \} \text{ e } C = \{ 3 \}$.

	Também temos:

	$A \cup B =\{ 2,4,5,6 \}$
	
	$A \cap B= \{ 6 \}$
	
	$A \cup C = \{ 2,3,4,6 \}$
	
	$A \cap C = B \cap C = \varnothing$
	
	$B \cup C = \{ 3,5,6 \}$.


	Também, $A^c = \{ 1,3,5\} , B^c = \{ 1,2,3,4 \} \text{ e } C^c = \{ 1,2,4,5,6\}$.

	Naturalmente, a álgebra de Boole aplicada a eventos pode ser usada em qualquer número deles. Assim,
em nosso caso, $A \cup B \cup C = \{2,3,4,5,6\} \text{ e } A \cap B \cap C = \varnothing$.
	
\end{exampleblock}

\end{frame}
%-----------------------------------------------------------------
\subsection{Probabilidades condicionais e conjuntas}
\begin{frame}

A probabilidade condicional de $B$ dado $A$

\begin{equation}
	P(B | A) = \frac{P(A \cap B)}{P(A)},\text{ se } P(A) > 0
\end{equation}


\begin{itemize}
	\item Se $X$ e $Y$ são independentes, temos:
	\begin{equation}
		P(x \cap y) = P(x)P(y)
	\end{equation}
	\item $P(x|y)$ é a probabilidade de $x$ dado $y$:
	\begin{equation}
		P(x|y) = \frac{P(x,y)}{P(y)}
	\end{equation}
	
	da mesma forma que

	\begin{equation}
		P(x,y) = P(x|y)P(y)
	\end{equation}
\end{itemize}

\end{frame}
%%-----------------------------------------------------------------
\subsection{O problema de Monty Hall}
\begin{frame}[c]


\end{frame}
%%-----------------------------------------------------------------
\begin{frame}[c]
	\textbf{PIC ou Arduino?}
	\newline

	\begin{center}
		\includegraphics[width=0.8\textwidth]{images/logo.jpg}
	\end{center}
\end{frame}
%-----------------------------------------------------------------

\begin{frame}[fragile]
	\frametitle{Exemplo codigo}
		Comando \textit{while}: Enquanto(Sentença):
		\begin{lstlisting}
			int i=0;
			while (i <= 10){
				...
				i=i+1;
			}
			i=0;
		\end{lstlisting}
			Comando \textit{for}: repita até(Sentença):
		\begin{lstlisting}language=C]
			int i=0;
			for (i=0;i<=10;i++){
				...
			}
		\end{lstlisting}
\end{frame}

%-----------------------------------------------------------------
\begin{frame}[fragile]
	\frametitle{Exemplo codigo}
	Comando \textit{while}: Enquanto(Sentença):

	\begin{lstlisting}[language=Python]
		import numpy as np
		
		def incmatrix(genl1,genl2):
			m = len(genl1)
			n = len(genl2)
			M = None #to become the incidence matrix
			VT = np.zeros((n*m,1), int)  #dummy variable
			#compute the bitwise xor matrix
			M1 = bitxormatrix(genl1)
			M2 = np.triu(bitxormatrix(genl2),1) 
			... 
	\end{lstlisting}
\end{frame}


\begin{frame}[fragile]

	\begin{lstlisting}[language=Python]
			... 
			for i in range(m-1):
					for j in range(i+1, m):
						[r,c] = np.where(M2 == M1[i,j])
						for k in range(len(r)):
							VT[(i)*n + r[k]] = 1;
							VT[(i)*n + c[k]] = 1;
							VT[(j)*n + r[k]] = 1;
							VT[(j)*n + c[k]] = 1;
							if M is None:
								M = np.copy(VT)
							else:
								M = np.concatenate((M, VT), 1)
							VT = np.zeros((n*m,1), int)
		return M
	\end{lstlisting}
\end{frame}

%-----------------------------------------------------------------
\begin{frame}[allowframebreaks]

\textbf{Referências}

\begin{itemize}
\begin{small}
\item PINHEIRO, Joao et al. Probabilidade e estatística: quantificando a incerteza. Elsevier Brasil, 2013.
\end{small}
\end{itemize}
\end{frame}
%-----------------------------------------------------------------
\end{document}
