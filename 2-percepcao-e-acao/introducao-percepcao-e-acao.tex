\documentclass{beamer}
\usetheme{simple} 
\usepackage[brazil]{babel}
\usepackage[utf8]{inputenc}
\usepackage{lmodern}
\usefonttheme[onlymath]{serif}
\usepackage[scale=2]{ccicons}
 
\usepackage{graphicx,hyperref,url,pgfplots}
\usepackage{amsmath} 
\usepackage{array,booktabs}
\pgfplotsset{compat=1.15} 

\setbeamercovered{invisible}
\newcommand{\pausar}{\pause}
\newcommand{\df}[1]{\,\mathrm{d}#1}
\newcommand{\parcial}[3]{\dfrac{\partial^{#1}#2}{\partial #3^{#1}}}

\usepackage{tikz}
\usepackage{xcolor}
\usetikzlibrary{scopes}
\usepackage{verbatim}
\usetikzlibrary{patterns}

\usepackage{listings}
	\definecolor{codegreen}{rgb}{0,0.6,0}
	\definecolor{codegray}{rgb}{0.5,0.5,0.5}
	\definecolor{codepurple}{rgb}{0.58,0,0.82}
	\definecolor{backcolour}{rgb}{0.92,0.92,0.92}
	\lstset{language=Python, 
	backgroundcolor=\color{backcolour},   
	commentstyle=\color{codegreen},
	keywordstyle=\color{magenta},
	numberstyle=\tiny\color{codegray},
	stringstyle=\color{codepurple},
	basicstyle=\fontsize{8}{11}\ttfamily,
	frame=lines,
%	numbers=left,
	tabsize=2,
	morekeywords={models, lambda, forms}}



% --------------------------------------------------------------------------------------------

\title{Percepção e Ação}
\subtitle{Sensores e Atuadores}
\date{\today}
\author[Jeferson José de Lima]{
  \textbf{Professor}: Jeferson José de Lima}
\institute[UTFPR-PB]{Departamento de Informática (DAINF)}

\begin{document}

\maketitle
 
\begin{frame}{Informações Úteis}
	\begin{block}{Material disponível em:}
		\href{Robótica Móvel - Wiki}{https://gitlab.com/cursoseaulas/robotica-movel/-/wikis/home}
	\end{block} 
	\pausar
	\begin{block}{Datas Importantes}
		\begin{itemize}
		\item Entrega
		\item Envio
		\end{itemize}
	\end{block}
	\pausar
	\begin{block}{Requisitos da Disciplina}
		\begin{itemize}
		\item Teoria de Controle
		\item Linguagem de Programação - \textbf{Python} ou \textbf{C++}
		\item Eletrônica
		\end{itemize}
	\end{block}
\end{frame}



\begin{frame}{Eventos Especiais}
União, interseção e complementação de eventos:

\begin{itemize}
	\item $A  \cup B$ é o evento que ocorre se (e somente se) pelo menos um dos eventos, $A$ ou $B$, ocorrer.
	\item $A \cap B$ é o evento que ocorre se ambos, $A$ e $B$, ocorrerem simultaneamente.
	\item $A^c$ , chamado evento complementar de $A$, é o evento cujos resultados pertencem a $\Omega$ mas não a $A$.
\end{itemize}

\begin{block}{Para}
	Um espaço amostral $\Omega$ associado a um experimento aleatório, e sejam $A$ e $B$ dois eventos
	contidos em $\Omega$. Diremos que $A$ e $B$ são mutuamente exclusivos se eles não possuem elementos comuns,
	isto é, se $A \cap B = \varnothing$.
\end{block}

\end{frame}



\begin{frame}
\begin{block}{Exemplo}

	Consideremos o lançamento de um dado equilibrado.
	
	Sejam $A= \{\text{número par}\} , B = \{\text{número maior que}\} \text{ e } C =\{ 3 \}$.
	
	Então $A =\{ 2,4,6 \} , B = \{ 5,6 \} \text{ e } C = \{ 3 \}$.

	Também temos:

	$A \cup B =\{ 2,4,5,6 \}$
	
	$A \cap B= \{ 6 \}$
	
	$A \cup C = \{ 2,3,4,6 \}$
	
	$A \cap C = B \cap C = \varnothing$
	
	$B \cup C = \{ 3,5,6 \}$.

	Também, $A^c = \{ 1,3,5\} , B^c = \{ 1,2,3,4 \} \text{ e } C^c = \{ 1,2,4,5,6\}$.

	Naturalmente, a álgebra de Boole aplicada a eventos pode ser usada em qualquer número deles. Assim,
em nosso caso, $A \cup B \cup C = \{2,3,4,5,6\} \text{ e } A \cap B \cap C = \varnothing$.
	
\end{block}
\end{frame}



\begin{frame}{Probabilidades condicionais e conjuntas}
\begin{itemize}
	\item A probabilidade condicional de $B$ dado $A$

	\begin{equation}
		P(B | A) = \frac{P(A \cap B)}{P(A)},\text{ se } P(A) > 0
	\end{equation}

	\item Partição de um espaço amostral

	Seja $B$ um evento qualquer do espaço amostral. Então os eventos $A_1 \cap B, A_2 \cap B,\cdots, A_m \cap B$, são todos
	mutuamente exclusivos e $B=\cup_{i=1}^{m}(A_i \cap B)$
	
	\item Probabilidade Total
	
	Se os eventos $A_1,A_2,\cdots,A_m$ formam uma partição do espaço amostral $\Omega$ e $B$ é um outro evento qualquer
	desse espaço, então:
	\begin{equation}
		P(B) = \sum\limits_{i=0}^{m}P(B|A_i)P(A_i)
	\end{equation}
\end{itemize}
\end{frame}



\begin{frame}
	Dizemos também, que a expressão $P(x_i, y_j) = P(X = x_i , Y = y_j)$ define a função de probabilidade conjunta da
	v.a. bidimensional discreta $( X, Y )$ se:

	a) $P(x_i, y_i) \geq 0 \text{ para todo par }(i,j)$
	
	b) $\sum\limits_{i,j} P(x_i, y_i) = 1 $

	\begin{itemize}
	\item Se $X$ e $Y$ os espaços amostrais, bem como, $x_i$, $y_i$ os eventos. Dado que $X$ e $Y$ são independentes, temos:
	\begin{equation}
		P(x,y) = P(x)P(y)
	\end{equation}
	\item $P(x|y)$ é a probabilidade de $x$ dado $y$:
	\begin{equation}
		P(x|y) = \frac{P(x,y)}{P(y)}
	\end{equation}
	
	da mesma forma que

	\begin{equation}
		P(x,y) = P(x|y)P(y)
	\end{equation}
\end{itemize}
\end{frame}


\begin{frame}[c]{Teorema de Bayes}
\end{frame}


\begin{frame}[c]{O problema de Monty Hall}
\end{frame}


\begin{frame}[t]{Referências}
    \begin{itemize}
        \item PINHEIRO, Joao et al. Probabilidade e estatística: quantificando a incerteza. Elsevier Brasil, 2013.
        
    \end{itemize}
\end{frame}
\end{document}