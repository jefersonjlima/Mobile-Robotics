\chapterauthor{Jeferson J. Lima}{Departamento de Informática (DAINF) \\Universidade Tecnológica Federal do Paraná (UTFPR)}
\chapter{Cinemática e Dinâmica de Robôs Móveis com Rodas}


\section{Introdução}\label{intro}
% 2. Cinemática e Dinâmica de Robôs Móveis com Rodas
% 	2.1 Cinemática Direta e Inversa
% 		2.1.1 Cinemática Direta e Inversa
% 		2.1.1 Cinemática de Robôs Móveis com Rodas
% 	2.2 Restrições de Movimento
% 		2.2.1 Sistemas Holonômicos
% 		2.2.2 Sistemas Não-Holonômicos
% 	2.3 Modelagem de Robô Móvel com restrições de Movimento.
% 3. Controle Moderno para Robótica Móvel
% 	3.1 Lyapunov-based Controle
% 	3.2 SDRE Controle
% 4. Eventos Não Deterministicos em Robótica Móvel
% 	4.1 Estimação de Estados
% 	4.2 Filtro Bayesiano
% 	4.3 Filtro de Kalman
% 		4.3.1 Filtro de Kalman Estendido

A humanidade é facinada pelo movimento das rodas a milhares de anos, isso intriga até os dias de hoje, o homem moderno.
Na matemática há areas expecíficas para descrever o movimento dos corpos, ou para o nosso interesse nesse capítulo, a cinemática dos robôs. 
Um modelo cinemático de um robô, com rodas ou pernas, descreve as relações geometricas entre que dão forma ao próposito deste robô.

Em resumo, os modelos cinemáticos estão relacionadas a como as velodicidades dos elementos do robô interagem entre si e o espaço.


\section{Cinemática Direta e Inversa}\label{intro-ch1}

A Cinemática é a ciência que trata do movimento (geométrico) e das forças que o causam [Craig]. Desta forma, dependendo do ponto de visão,
podemos referênciar a cinemática de um robô atravês de uma referência externa (Cinemática externa), como por exemplo a relação entre o robô e
as coordenadas de um mapa global. Há a possibilidade também, da referência ser o próprio robô, desta forma da-se o nome de
Cinemática Interna, como exemplo, podemos citar a velocidade de rotação em torno do seu eixo ou velocidade das rodas.

Podemos ainda nos aprofundar mais sobre a Cinemática Interna do Robô, onde o espaço de ação das coordenadas definem a forma que se deve tratar o modelo de equações.
Considerando um robô fixo ou móvel com coodenadas generalizadas $\theta_1, \theta_2,..., \theta_n$ localizadas no espaço das \textcolor{red}{juntas ou atuadores (\textit{joint space}) $\mathbf{q}$}. Bem como $x_1, x_2,..., x_n$, o \textcolor{blue}{espaço das tarefas (\textit{task space}) $\mathbf{x}$}, temos então os vetores:

\begin{figure}[!ht]


\begin{tikzpicture}
    \newcommand{\nvar}[2]{%
    \newlength{#1}
    \setlength{#1}{#2}
    }

    % Define a few constants for drawing
    \nvar{\dg}{0.3cm}
    \def\dw{0.25}\def\dh{0.5}
    % Define commands for links, joints and such
    \def\link{\draw [double distance=1.5mm, very thick] (0,0)--}
    \def\joint{%
    \filldraw [fill=white] (0,0) circle (5pt);
    \fill[black] circle (2pt);
    }
    \def\grip{%
    \draw[ultra thick, blue](0cm,\dg)--(0cm,-\dg);
    \fill[blue] (0cm, 0.5\dg)+(0cm,1.5pt) -- +(0.6\dg,0cm) -- +(0pt,-1.5pt);
    \fill[blue] (0cm, -0.5\dg)+(0cm,1.5pt) -- +(0.6\dg,0cm) -- +(0pt,-1.5pt);
    }

    \def\robotbase{%
    \draw[rounded corners=8pt] (-\dw,-\dh)-- (-\dw, 0) --
        (0,\dh)--(\dw,0)--(\dw,-\dh);
    \draw (-0.5,-\dh)-- (0.5,-\dh);
    \fill[pattern=north east lines] (-0.5,-1) rectangle (0.5,-\dh);
    }
    \newcommand{\doublelink}[6]{%
    \robotbase
    \link(#1:#2);
    \joint
    \node[left]{$\color{black}{\theta_1}$};
    \begin{scope}[shift=(#1:#2), rotate=#1]
        \link(#3:#4);
        \joint
        \node[above]{$\color{black}{\theta_2}$};
        \begin{scope}[shift=(#3:#4), rotate=#5]
            \grip
            \node[right]{$\color{blue}{\mathbf{x}_{tool}}$};
        \end{scope}
    \end{scope}
    }

    \doublelink{60}{2}{-90}{2}{-60}{1}
\end{tikzpicture}
    
\caption{Robô - Dois Graus de Liberdade}
\label{fig:2dof-robot}
\end{figure}

Assim sendo, da-se o nome de a \textcolor{red}{Cinemática Direta} quando o robô e descrito como função de entradas como (velocidade das rodas, movimento das juntas, direção das rodas).  Já a \textcolor{blue}{Cinemática Inversa} possibilita projetar um planejamento de movimento, o que significa que as entradas do robô podem ser calculadas para uma sequência de estado do robô desejada.

A relação entre as Cinemática Direta e Cinemática Inversa é obtida através da Matriz Jacobiana do Robô.

\begin{equation*}
    \mathbf{\dot{x}} = \mathbb{J}{\mathbf{\dot{q}}}
    \text{ e, }
    \mathbf{\dot{q}} = \mathbb{J}^{-1}{\mathbf{\dot{x}}}
\end{equation*}

bem como:

\begin{equation*}
    \frac{\text{d}\mathbf{x}}{\text{d}t} = \mathbb{J}\frac{\text{d}\mathbf{q}}{\text{d}t}
    \text{ e, }
    \frac{\text{d}\mathbf{q}}{\text{d}t} = \mathbb{J}^{-1}\frac{\text{d}\mathbf{x}}{\text{d}t}
\end{equation*}

onde $\mathbb{J}$ é dado por:
\begin{equation*}
    \mathbb{J}
    =
    \frac{d \mathbf{f}}{d \mathbf{q}}
    =
    \left[ \frac{\partial \mathbf{f}}{\partial q_1}
        \cdots \frac{\partial \mathbf{f}}{\partial q_n} \right]
    =
    \begin{bmatrix}
        \frac{\partial f_1}{\partial q_1} & \cdots &
        \frac{\partial f_1}{\partial q_n}                   \\
        \vdots                            & \ddots & \vdots \\
        \frac{\partial f_m}{\partial q_1} & \cdots &
        \frac{\partial f_m}{\partial q_n}
    \end{bmatrix}
\end{equation*}

Essa transformação é feita quando necessita-se, por exemplo, controlar um robô utilizando-se das referências de uma ferramenta acoplada a
extremidade do braço robótico. A Cinemática Inversa proporciona que a estratégia de controle seja aplicada na ferramenta até
que o objetivo da tarefa seja alcançado.


\begin{shortbox}
    \Boxhead{Exercício de Fixação}
    Levando em consideração o braço robótico da Figura \ref{fig:2dof-robot}, encontre as equações da cinemática do robô.
    \begin{equation*}
        \begin{split}
            x_f(t) = & l_1\cos(\theta_1)+l_2\cos(\theta_1 + \theta_2) \\            
            y_f(t) = & l_1\sin(\theta_1)+l_2\sin(\theta_1 + \theta_2)
        \end{split}
    \end{equation*}
    
    utilize a biblioteca \textbf{sympy} para solução do jacobiano.
    \begin{center}
        Resposta:

        \qrcode[height=0.5in]{https://ocw.mit.edu/courses/mechanical-engineering/2-12-introduction-to-robotics-fall-2005/lecture-notes/chapter5.pdf}
\end{center}

\end{shortbox} 


\section{Cinemática de Robôs Móveis com Rodas}

\subsection{Cinemática Biciclo}
\subsection{Cinemática Robô Diferencial}
\subsection{Robô Omnidirecional}


\section{Dinâmica com Restrições de Movimento}
\subsection{Sistemas Holonômicos}
\subsection{Sistemas Não-Holonômicos}

\section{Modelagem de Robô Móvel com restrições de Movimento}



\section{Glossary}
\begin{Glossary}
\item[360 Degree Review] Performance review that includes feedback from superiors, peers, subordinates, and clients.
\item[Abnormal Variation] Changes in process performance that cannot be accounted for by typical day-to-day variation. Also referred to as
non-random variation.
\item[Acceptable Quality Level (AQL)] The minimum number of parts that must comply with quality standards, usually stated as a percentage.
\item[Activity] The tasks performed to change inputs into outputs.
\item[Adaptable] An adaptable process is designed to maintain effectiveness and efficiency as requirements change. The process is
deemed adaptable when there is agreement among suppliers, owners, and customers that the process will meet
requirements throughout the strategic period.
\end{Glossary}



